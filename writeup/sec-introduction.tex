\section{Introduction}

The discrete Fourier transform (DFT) is an orthogonal linear
transformation that serves as a discrete analog of the Fourier
transform. Many problems in science and engineering yield to the
methods of Fourier analysis, so much so that when the fast Fourier
transform (FFT) algorithm of Cooley and Tukey was disclosed in 1965,
much of the focus of research in the field of signal processing
shifted from research in analog signal processing to digital signal
processing, owing to the newfound feasibility of previously
speculative numerical methods~\cite{book:dsp75}. Of course, this
algorithm has gone on to enable the efficient solution of a myriad of
problems since.

A limitation of the $N$-point
FFT is that its domain consists of vectors in $\mathbb{C}^N$
whose components are typically assumed to correspond to equally spaced
samples of some underlying continuous signal; likewise, the range is
comprised of coefficients of equally spaced frequency components. In
the cases where our input signal is sampled unevenly or where we would
like to determine the strength of frequency components which are
unequally spaced, the usual FFT is of no use without interpolating
before or after the transform. A variety of methods have been
introduced to formalize an approach to computing a ``nonequispaced''
discrete Fourier transform making use of
interpolation~\cite{Dutt95fastfourierII,Keiner06nfft3.0}.

In this project, our goal is to explore the practicality of a new
approach to computing the nonuniform IDFT (INDFT) efficiently. Like
these other methods, the approach relies on interpolation, and this
project will specifically consider a fast INDFT (INFFT).

This paper is structured as follows:
\begin{enumerate}
\item We outline basic facts related to the sampling theorem that
  will serve as an important step in the derivation of our algorithm.
\item We pose the interpolation problem as a matrix product and
  factorize this product into a form amenable to solution by a method
  for computing periodic sums~\cite{Gumerov13amethod}, which makes use
  of the fast multipole method (FMM).
\item We derive the corresponding $S$ and $R$ factorizations of the
  relevant kernel for use in the FMM.
\item Likewise, we compute the $\SSmat$, $\SRmat$, and $\RRmat$ translation
  operators for this kernel.
\item We derive the periodic summation algorithm.
\item The inverse interpolation algorithm is presented.
\item The test series used in the project are considered.
\item We discuss our numerical experiments.
\item We look into future work to be conducted, and treat in more
  depth the optimizations that could be made to our implementation of
  the FMM.
\end{enumerate}