\section{Computation of $\persumcoef{0}$}

The first column of the fitting matrix $\m{R}$ is zero, which prevents
the preceding least squares fitting procedure from recovering
$\persumcoef{0}$. For some problems, this does not present a problem,
but for this one it does, and a method of computing $\persumcoef{0}$
is necessary. As it happens, approximating $\persumcoef{0}$ from:
\begin{align}
  \label{eq:c0-direct}
  \persumcoef{0} = \sum_{\persumindex\notin\persumdomain}\sum_{\unifptindex=0}^{\bandlimit-1} \frac{{(-1)}^k \unifval{\unifptindex}}{\unifpt{\unifptindex} + 2\pi(\persumindex - 1)}
\end{align}
is unrealistic due to slow convergence---the series is similar to the
alternating harmonic series, for instance. Other methods for more
general problems have been suggested~\cite{periodic-sums}.

In our case, we find that, for each $y\in\R$:
\begin{align}\label{eq:c0-est}
  c_0 = \frac{1}{\bandlimit} \sum_{\unifptindex=0}^{\bandlimit-1} \phinear(y - \unifpt{\unifptindex}).
\end{align}
Using the FMM, we are able to approximately compute $c_0$ without
compromising the overall complexity of the algorithm. What's more, by
merging each $y - \unifpt{\unifptindex}$ into the set of target points
and using $y - \unifpt{\unifptindex}$ for the algorithm's checkpoints,
$c_0$ can be estimated without compromising the performance of the
implementation. An added benefit of this approach is that, since $c_0$
computed this way does not depend on the set of target points, it can
be precomputed so that the cost of its computation is amortized over
repeated invocations.

As a matter of practical importance, we also note that $\phinear$ as
computed by the FMM is most accurate at the abscissae
$(2\unifptindex + 1)\pi{}/\bandlimit$, for
$k = 0, \hdots, \bandlimit - 1$. An implementation of the algorithm
presented in this paper, then, should select $y = \pi/\bandlimit$ when
computing $c_0$ in the above fashion.

If the overhead of computing the FMM at $\bandlimit$ points is too
much, another approach which provides comparable (in our experiments,
indistinguishable) accuracy is by bounding the series defining $c_0$
in the same fashion as in the proof of convergence of $c_0$ in the
proof of the necessary condition for the periodic summation
method. Specifically, we have that $c_0$ is approximately given by:
\begin{align}
  \label{eq:c0-arctanh}
  c_0 \approx \frac{1}{2\pi} \dftsum{\unifptindex{}} {(-1)}^\unifptindex \unifval{\unifptindex} \cdot \parens{\arctanh\parens{\frac{\unifpt{\unifptindex}-\pi}{2\pi\neighborhoodradius}} + \arctanh\parens{\frac{\unifpt{\unifptindex}-\pi}{2\pi(\neighborhoodradius + 1)}}}.
\end{align}
This quantity can be computed in $O(\bandlimit)$ time, and in our
numerical experiments we found that it gave results as accurate as the
results of the previous method. Because of this, this is the method
used in our implementation.

%%% Local Variables:
%%% mode: latex
%%% TeX-master: "../paper.tex"
%%% End:
