\section{The Nonuniform FFT}

Let $\bandfunc: \R \to \R$ be a bandlimited function, and let
$\bandlimit{} \in \N$ be its bandlimit. If, for each
$\unifptindex{} \suchthat 0 \leq \unifptindex{} < \bandlimit{}$, we
let $\unifpt{\unifptindex{}} = 2\pi{}\unifptindex{}/\bandlimit{}$ and
$\unifval{\unifptindex{}} = \bandfunc(\unifpt{\unifptindex{}})$, then
the DFT of $\unifval{0}, \hdots, \unifval{\bandlimit{} - 1}$ is given
by:
\begin{align}\label{eq:dft}
  \unifdft{\unifdftindex{}} = \sum_{\unifptindex{}=0}^{\bandlimit{}-1} \unifval{\unifptindex{}} e^{-2\pi{}i\unifptindex{}\unifdftindex{}/\bandlimit{}}
\end{align}
for $\unifdftindex{} \suchthat 0 \leq \unifdftindex{} <
\bandlimit{}$. The $\unifval{\unifptindex{}}$'s and
$\unifdft{\unifdftindex{}}$'s can be thought of as ordinates
associated with corresponding equispaced ascissae. For arguments off
of these regular grids of points (in the time and frequency domains,
respectively), the question as to how to evaluate $f$ naturally
arises. The general class of solutions provides methods for computing
what is referred to as the nonuniform DFT.\@

The inverse DFT formula leads us to interpret the DFT coefficients
$\unifdft{0}, \hdots, \unifdft{\bandlimit{}-1}$ as the coefficients of
a trigonometric polynomial:
\begin{align}\label{eq:idft-polynomial}
  \bandfunc(x) = \sum_{\unifdftindex{}=-\bandlimit{}/2+1}^{\bandlimit{}/2-1} \unifdft{\unifdftindex{}} e^{i\unifdftindex{}x}
\end{align}
If, for $\arbptindex{} \suchthat 0 \leq \arbptindex{} < \numarbpts{}$, we define
$\arbpt{\arbptindex{}} \suchthat 0 \leq \arbpt{\arbptindex{}} < 2\pi$, then combining
(\ref{eq:dft}) with (\ref{eq:idft-polynomial}) evaluated at each
$\arbpt{\arbptindex{}}$ yields:
\begin{align}\label{eq:first-interpolation}
  \arbval{\arbptindex{}} \defd \bandfunc(\arbpt{\arbptindex{}}) = \frac{1}{\bandlimit{}} \sum_{\unifptindex{}=0}^{\bandlimit{}-1} \parens{\sum_{\unifdftindex{}=-\bandlimit{}/2+1}^{\bandlimit{}/2-1} e^{i\unifdftindex{}(\arbpt{\arbptindex{}} - \unifpt{\unifptindex{}})}} \unifval{\unifptindex{}}.
\end{align}
It is readily shown that this expression can be manipulated further to
yield:
\begin{align}\label{eq:per-interp-cot-kernel}
  \arbval{\arbptindex{}} = \frac{\sin(\bandlimit{}\arbpt{\arbptindex{}}/2)}{\bandlimit{}} \sum_{\persumindex{}=-\infty}^\infty \sum_{\unifptindex{}=0}^{\bandlimit{}-1} \frac{{(-1)}^{\unifptindex{}}\unifval{\unifptindex{}}}{\arbpt{\arbptindex{}} - \unifpt{\unifptindex{}} - 2\pi{}\persumindex{}}.
\end{align}
A proof of this can be found in the appendix.

Equation (\ref{eq:per-interp-cot-kernel}) allows us to consider the
interpolation as a linear operator applied to the values of a fourier
series at equispaced points. We can define this operator
$\interpop\in\R^{\numarbpts\times\bandlimit}$ by:
\begin{align}
  \label{eq:interp-operator}
  \interpop_{\arbptindex,\unifptindex} \defd \frac{{(-1)}^\unifptindex \sin(\bandlimit\arbpt{\arbptindex}/2)}{\bandlimit} \sum_{\persumindex=-\infty}^\infty \frac{1}{\arbpt{\arbptindex} - \unifpt{\unifptindex} - 2\pi\persumindex}.
\end{align}
Thus, letting $\unifvalvec\in\R^\bandlimit$ and
$\arbvalvec\in\R^\numarbpts$ be defined by
$\unifvalvec_\unifptindex = \unifval{\unifptindex}$ and
$\arbvalvec_\arbptindex = \arbval{\arbptindex}$, respectively, we can
see that $\arbvalvec = \interpop\unifvalvec$. The expression for
$\arbval{\arbptindex{}}$ provided in (\ref{eq:per-interp-cot-kernel})
can be derived from results in the paper from which this work
primarily derives~\cite{Dutt95fastfourierII}. The inverse
interpolation formula is also given in this paper. From this formula,
we have that the nominal approximate ``inverse'' interpolation
operator $\invinterpop\in\R^{\bandlimit\times\numarbpts}$ is defined
by letting:
\begin{align}
  \ccoef{\arbptindex} &\defd \prod_{\unifptindex=0}^{\bandlimit-1} \sin\parens{\frac{\arbpt{\arbptindex}-\unifpt{\unifptindex}}{2}}, \label{eq:cjs} \\
  \dcoef{\unifptindex} &\defd \prod_{\substack{\unifptindex'=0 \\ \unifptindex'\neq\unifptindex}}^{\bandlimit-1} \frac{1}{\sin\parens{\frac{\unifpt{\unifptindex} - \unifpt{\unifptindex{}'}}{2}}}, \label{eq:dks} \\
  \invinterpop_{\unifptindex,\arbptindex} &\defd \ccoef{\arbptindex}\cdot\dcoef{\unifptindex}\cdot\sum_{\persumindex=-\infty}^\infty \frac{1}{\arbpt{\arbptindex} - \unifpt{\unifptindex} - 2\pi\persumindex}. \label{eq:inverse-interp-operator}
\end{align}
The corresponding formula for recovering $\unifval{\unifptindex}$ from
$\arbval{\arbptindex}$ is:
\begin{align}\label{eq:per-inv-interp-cot-kernel}
  \unifpt{\unifptindex} = \dcoef{\unifptindex} \sum_{\persumindex=-\infty}^\infty\sum_{\arbptindex=0}^{\numarbpts-1}\frac{\ccoef{\arbptindex}}{\arbpt{\arbptindex} - \unifpt{\unifptindex} - 2\pi\persumindex}.
\end{align}

If we let $\fft$ denote a linear operator corresponding to the DFT,
e.g.\ an application of some FFT, then there are four operators of
interest which together comprise the nonuniform FFT
(Figure~\ref{fig:operators}).  Of course, further combinations are
possible---e.g.\ $\invtransinterpop\circ\fft\circ\invinterpop$. As
mentioned, the goal is to the design algorithms which compute
$\interpop$ and its variations with complexity no worse than $\fft$
and $\ifft$. As it happens, each of
(\ref{eq:interp-operator}--\ref{eq:inverse-interp-operator}) can be
computed sufficiently fast. Writing:
\begin{align}
  \label{eq:log-sin}
  \log(\ccoef{\arbptindex}) = \sum_{\arbptindex=0}^{\bandlimit-1} \log\circ\sin\parens{\frac{\arbpt{\arbptindex}-\unifpt{\unifptindex}}{2}},
\end{align}
and similarly for $\dcoef{\unifptindex}$, allows the FMM for the
$\log\circ\sin$ kernel to be applied so that these coefficients may be
precomputed, although this is not our focus.

\begin{figure}[h]
  \centering
  \begin{tabular}[h]{ll}
    Operator & Description \\
    \midrule
    $\nufft \defd \interpop\circ\ifft$ & Compute the IDFT and interpolate in the time domain. \\
    $\nuffttrans \defd \ifft\circ\transinterpop$ & Interpolate uniform frequencies from arbitrary ones and compute IDFT.\@ \\
    $\nufftinv \defd \fft\circ\invinterpop$ & Approximate uniformly spaced Fourier series values and compute the DFT.\@ \\
    $\nufftinvtrans \defd \invtransinterpop\circ\fft$ & Apply the DFT and interpolate in the frequency domain. \\
  \end{tabular}
  \caption{the various nonuniform FFT operators.}\label{fig:operators}
\end{figure}

% Local Variables:
% TeX-master: "../paper.tex"
% indent-tabs-mode: nil
% End:
