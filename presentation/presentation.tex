\documentclass{beamer}

\usepackage{
  amsmath,
  amssymb,
  amsthm,
  tikz,
  xcolor
}

\newcommand{\abs}[1]{\card{#1}}
\newcommand{\asconv}{\overset{\operatorname{as}}{\longrightarrow}}
\newcommand{\bern}[1]{\operatorname{Bernoulli} \parens{#1}}
\newcommand{\bigcdot}{\boldsymbol{\cdot}}
\newcommand{\boldup}[1]{\textbf{\textup{#1}}}
\newcommand{\card}[1]{\left|#1\right|}
\newcommand{\comp}[1]{{#1}^{c}}
\newcommand{\condexpected}[2]{E \left[ #1 \;\middle\vert\; #2 \right]}
\newcommand{\condprob}[2]{P \left[ #1 \;\middle\vert\; #2 \right]}
\newcommand{\cov}[2]{\operatorname{cov} \parens{#1, #2}}
\newcommand{\curlyb}[1]{\left\{#1\right\}}
\newcommand{\distconv}{\overset{\operatorname{d}}{\longrightarrow}}
\newcommand{\expected}[1]{E \squareb{#1}}
\newcommand{\exponential}[1]{\operatorname{Exp} \parens{#1}}
\newcommand{\geom}[1]{\operatorname{Geometric} \parens{#1}}
\newcommand{\icoms}{\longleftrightarrow}
\newcommand{\io}{\mbox{i.o.}}
\newcommand{\nats}{\mathbb{N}}
\newcommand{\probarc}[1]{\stackrel{#1}{\longrightarrow}}
\newcommand{\parens}[1]{\left(#1\right)}
\newcommand{\poisson}[1]{\mbox{Poisson} \parens{#1}}
\newcommand{\powerset}[1]{2^{#1}}
\newcommand{\probconv}[1]{\overset{#1}{\longrightarrow}}
\newcommand{\prob}[1]{P \squareb{#1}}
\newcommand{\qmconv}{\overset{\operatorname{qm}}{\longrightarrow}}
\newcommand{\range}{\operatorname{range}}
\newcommand{\reaches}{\longrightarrow}
\newcommand{\set}[1]{\left\{#1\right\}}
\newcommand{\squareb}[1]{\left[#1\right]}
\newcommand{\uniform}[2]{\operatorname{Uniform}\left(#1, #2\right)}
\newcommand{\var}[1]{\operatorname{var} \parens{#1}}

% commands for this paper

\newcommand{\phinear}{\phi_{\operatorname{near}}}
\newcommand{\phifar}{\phi_{\operatorname{far}}}
\newcommand{\SSmat}{\bold{S}|\bold{S}}
\newcommand{\SRmat}{\bold{S}|\bold{R}}
\newcommand{\RRmat}{\bold{R}|\bold{R}}

% title etc

\title{FMM-based Bandlimited Interpolation}
\author{Sam Potter}

\begin{document}

\setbeamertemplate{itemize items}[triangle]

\frame{\titlepage}

\begin{frame}
  \frametitle{Motivation}
  \begin{itemize}
  \item The FFT and IFFT are a pair of algorithms that convert in
    $O(N \log N)$ time between:
    \begin{align*}
      \set{f(x_k)} \underset{\operatorname{FFT}}{\overset{\operatorname{IFFT}}{\leftrightarrows}} \set{c_n},
    \end{align*}
    where $x_k = 2\pi k/N$ and:
    \begin{align*}
      c_n = {1 \over N} \sum_{k=0}^{N-1} f(x_k) e^{-inx_k}.
    \end{align*}
  \end{itemize}
\end{frame}

\begin{frame}
  \frametitle{Motivation}
  \begin{itemize}
  \item The input and output of the FFT are thought of as equispaced
    grids of point in space or time and frequency, respectively. What
    if we don't have equispaced input or don't want equispaced output?
    \pause
  \item We need to interpolate.
  \end{itemize}
\end{frame}

\begin{frame}
  \frametitle{Setup}
  \begin{itemize}
  \item There are six possible algorithms:
    \begin{align*}
      \begin{tabular}{ccc}
        space/time domain & & frequency domain \\
        equispaced & $\leftrightarrows$ & nonequispaced \\
        nonequispaced & $\leftrightarrows$ & equispaced \\
        nonequispaced & $\leftrightarrows$ & nonequispaced
      \end{tabular}
    \end{align*}
  \item This project focuses on this case:
    \begin{align*}
      \begin{tabular}{ccc}
        \textbf{nonequispaced} & $\boldsymbol{\leftarrow}$ & \textbf{equispaced}
      \end{tabular}      
    \end{align*}
  \end{itemize}
\end{frame}

\begin{frame}
  \frametitle{Setup}
  \begin{itemize}
  \item Assume that $f$ is \emph{bandlimited} so that:
    \begin{align*}
      f(x) = \sum_{n=-(N-1)}^{N-1} c_n e^{jnx}
    \end{align*}
  \item The function $f$ is $2\pi$-periodic.
  \item Let there be $M$ points $y_j$ distributed in $[0, 2\pi)$.
  \item And let $g_j = f(y_j)$.
  \end{itemize}
\end{frame}

\begin{frame}
  \frametitle{Setup}
  \begin{itemize}
  \item A little algebra and the fact that $f$ is bandlimited gives us:
    \begin{align*}
      g_j = f(y_j) = \sum_{k=0}^{N-1} f(x_k) \parens{{1 \over N} \sum_{n=-(N-1)}^{N-1} e^{-inx_k} e^{iny_j}}
    \end{align*}
    \pause
  \item \ldots suggesting that we define a matrix $\bold{K}$ by:
    \begin{align*}
      K_{jk} = {1 \over N} \sum_{n=-(N-1)}^{N-1} e^{-inx_k} e^{iny_j}
    \end{align*}
  \end{itemize}
\end{frame}

\begin{frame}
  \frametitle{Setup}
  \begin{itemize}
  \item Stacking the $f(x_k) = f_k$'s
    and $g_j$'s into the vectors $\bold{f}$ and $\bold{g}$ lets us write:
    \begin{align*}
      \bold{g} = \bold{K} \bold{f}
    \end{align*}
  \item Solving our interpolation problem is the same as solving for
    $\bold{g}$.
    \pause
  \item \textbf{Bad:} naive matrix multiplication has complexity $O(N^2)$.
    \pause
  \item \textbf{Worse:} our resulting IFFT algorithm is $O(N^2)$.
    \begin{align*}
      \underbrace{\set{f(x_k)} \; \overset{O(N^2)}{\longleftarrow} \; \set{g_j} \overset{O(N \log N)}{\longleftarrow} \set{c_n}}_{O(N^2)}
    \end{align*}
  \end{itemize}
\end{frame}

\begin{frame}
  \frametitle{Setup}
  \begin{itemize}
  \item We can rewrite the entries of $\bold{K}$ as:
    \begin{align*}
      K_{jk} = {1 \over N} \parens{-1 + {e^{iNy_j} - 1 \over e^{i(y_j - x_k)} - 1} + {e^{-iNy_j} - 1 \over e^{i(x_k - y_j)} - 1}}
    \end{align*}
    \pause
  \item From this expression, we define:
    \begin{align*}
      F_j^+ = {e^{iNy_j} - 1 \over N} \hspace{1cm} F_j^- = {e^{-iNy_j} - 1 \over N}
    \end{align*}
    \begin{align*}
      G(t) = {1 \over e^{it} - 1}
    \end{align*}
  \end{itemize}
\end{frame}

\begin{frame}
  \frametitle{Setup}
  \begin{itemize}
  \item For $G(t) = (e^{it} - 1)^{-1}$, we have:
    \begin{align*}
      G(t) = {1 + i \cot(t/2) \over 2}
    \end{align*}
  \item We also have:
    \begin{align*}
      \cot(t) = \sum_{k=-\infty}^\infty {1 \over t - \pi k}
    \end{align*}
    \pause
  \item Combining these yields:
    \begin{align*}
      G(t) = {1 \over 2} + i \sum_{k=-\infty}^\infty {1 \over t - 2\pi k}
    \end{align*}
  \end{itemize}
\end{frame}

\begin{frame}
  \frametitle{Setup}
  \begin{itemize}
  \item Altogether, for the components of $\bold{g}$, we can write:
    \begin{align*}
      g_j = &\parens{{F_j^+ + F_j^- \over 2} - N^{-1}} \sum_{k=0}^{N-1} f_k \; + \\ &i \parens{F_j^+ - F_j^-} \sum_{l=-\infty}^\infty \sum_{k=0}^{N-1} {f_k \over y_j - x_k - 2\pi l}
    \end{align*}
    \pause
  \item \textbf{Goal}: combine this equation with the fast multipole
    method to compute $\bold{g} = \bold{K} \bold{f}$
    in $O(N \log N)$ time or better.
  \end{itemize}
\end{frame}

\begin{frame}{A Brief Detour}
  \begin{itemize}
  \item Notice that the highlighted term is an \emph{infinite,
      $2\pi$-periodic summation of simple kernels that we've seen in class}:
    \begin{align*}
      g_j = & \parens{{F_j^+ + F_j^- \over 2} - N^{-1}} \sum_{k=0}^{N-1} f_k \; + \\ & i \parens{F_j^+ - F_j^-} \textcolor{red}{\sum_{l=-\infty}^\infty \sum_{k=0}^{N-1} {f_k \over y_j - x_k - 2\pi l}}
    \end{align*}
  \item We'll summarize the relevant factorizations and translation operators.
  \end{itemize}
\end{frame}

\begin{frame}{A Brief Detour}{Kernel Factorization}
  \begin{itemize}
  \item We define the kernel $\Phi$ by:
    \begin{align*}
      \Phi(y, x) = {1 \over y - x}.
    \end{align*}
    \pause
  \item Its regular factorization is given by:
    \begin{align*}
      \Phi(y, x) = {1 \over x - x_*} \sum_{m=0}^\infty \parens{y - x_* \over x - x_*}^m = \sum_{m=0}^\infty a_m(x, x_*) R_m(y - x_*).
    \end{align*}
    \pause
  \item And its singular factorization by:
    \begin{align*}
      \Phi(y, x) = {1 \over y - x_*} \sum_{m=0}^\infty \parens{x - x_* \over y - x_*}^m = \sum_{m=0}^\infty b_m(x, x_*) S_m(y - x_*)
    \end{align*}
  \end{itemize}
\end{frame}

\begin{frame}{A Brief Detour}{$\bold{S}|\bold{S}$ Translation Operator}
  \begin{itemize}
  \item The $\bold{S}|\bold{S}$ operator is given by:
    \begin{align*}
      \bold{S}|\bold{S} = \begin{bmatrix}
        1 & 0 & 0 & 0 & 0 & \cdots \\
        -\delta & 1 & 0 & 0 & 0 & \cdots \\
        \delta^2 & -2\delta & 1 & 0 & 0 & \cdots \\
        -\delta^3 & 3\delta^2 & -3\delta & 1 & 0 & \cdots \\
        \delta^4 & -4\delta^3 & 6 \delta^2 & -4 \delta & 1 & \cdots \\
        \vdots & \vdots & \vdots & \vdots & \vdots & \ddots
      \end{bmatrix}
    \end{align*}
    \pause
  \item Great structure\----lower triangular, alternating signs, ascending powers of $\delta$, binomial coefficients\ldots
  \item Should be possible to write a very efficient algorithm to compute this operator.
  \end{itemize}
\end{frame}

\begin{frame}{A Brief Detour}{$\bold{S}|\bold{R}$ Translation Operator}
  \begin{itemize}
  \item Likewise, the $\bold{S}|\bold{R}$ matrix is written:
    \begin{align*}
      \bold{S}|\bold{R} = \begin{bmatrix}
        \delta^{-1} & \delta^{-2} & \delta^{-3} & \delta^{-4} & \cdots \\
        -\delta^{-2} & -2\delta^{-3} & -3\delta^{-4} & -4\delta^{-5} & \cdots \\
        \delta^{-3} & 3\delta^{-4} & 6 \delta^{-5} & 10 \delta^{-6} & \cdots \\
        -\delta^{-4} & -4 \delta^{-5} & -10 \delta^{-6} & -20 \delta^{-7} & \cdots \\
        \vdots & \vdots & \vdots & \vdots & \ddots
      \end{bmatrix}
    \end{align*}
  \end{itemize}
\end{frame}

\begin{frame}{A Brief Detour}{$\bold{R}|\bold{R}$ Translation Operator}
  \begin{itemize}
  \item And the $\bold{R}|\bold{R}$ matrix is written:
    \begin{align*}
      \bold{R}|\bold{R} = \begin{bmatrix}
        1 & \delta & \delta^2 & \delta^3 & \delta^4 & \cdots \\
        0 & 1 & 2\delta & 3 \delta^2 & 4\delta^3 & \cdots \\
        0 & 0 & 1 & 3 \delta & 6\delta^2 & \cdots \\
        0 & 0 & 0 & 1 & 4 \delta & \cdots \\
        0 & 0 & 0 & 0 & 1 & \cdots \\
        \vdots & \vdots & \vdots & \vdots & \vdots & \ddots
      \end{bmatrix}
    \end{align*}
  \end{itemize}
\end{frame}

\begin{frame}{Computing the Periodic Summation}
  \begin{align*}
    \phi(y) = \sum_{l=-\infty}^\infty \sum_{k=0}^{N-1} {f_k \over y - x_k + 2 \pi l}
  \end{align*}
  \begin{itemize}
  \item We can compute this summation using a method presented in a
    paper by Professors Gumerov and Duraiswami (\emph{A method to
      compute periodic sums}, 2014, Journal of Computational Physics).
    \pause
  \item We will present an adaption of their algorithm which applies to
    our particular summation.
  \end{itemize}
\end{frame}

\begin{frame}{Computing the Periodic Summation}{Dividing the Domain}
  \begin{figure}[h]
    \centering
    \begin{tikzpicture}
      \draw[-] (-0.6,0) -- (2.6, 0);
      \draw[-] (0, 0.1) -- (0, -0.1) node[below] {$0$};
      \fill[black] (1, 0) circle (1pt);
      \draw[dashed] (1, 0) -- (1, -1) node[below] {$x_* = \pi$};
      \draw[-] (2, 0.1) -- (2, -0.1) node[below] {$2\pi$};
      
      \fill[black] (-0.85, 0) circle (0.5pt);
      \fill[black] (-1.0, 0) circle (0.5pt);
      \fill[black] (-1.15, 0) circle (0.5pt);

      \fill[black] (2.85, 0) circle (0.5pt);
      \fill[black] (3.0, 0) circle (0.5pt);
      \fill[black] (3.15, 0) circle (0.5pt);

      \draw[->] (-1.4, 0) -- (-2.25, 0);
      \draw[-] (-2, 0.1) -- (-2, -0.1) node[below] {$-2\pi n$};

      \draw[->] (3.4, 0) -- (4.25, 0);
      \draw[-] (4, 0.1) -- (4, -0.1) node[below] {$2\pi (n + 1)$};

      \draw[dashed] (1, 1) -- (3.7, 1);
      \draw[-] (1, 1.1) -- (1, 0.9);
      \draw[-] (2.7, 1.3) node[above] {$r + \pi$} -- (2.7, 0.9);
      \draw[-] (3.32, 1.1) -- (3.32, 0.7) node[below] {$R + \pi$};
    \end{tikzpicture}
  \end{figure}
  \begin{itemize}
  \item Choose $n \geq 1$.
  \item Pick $r$ and $R$ so that:
    \begin{align*}
      \pi < r < R < 2\pi(n + 1)
    \end{align*}
  \end{itemize}
\end{frame}

\begin{frame}{Computing the Periodic Summation}{Near and Far Components}
  \begin{itemize}
  \item Define $\mathcal{N} = \set{-n, -n+1, \hdots, n} \subset \mathbb{Z}$.
    \pause
  \item Decompose $\phi$ by defining $\phinear$ and $\phifar$ by:
    \begin{align*}
      \phinear(y) &= \sum_{l \in \mathcal{N}} \sum_{k=0}^{N-1} {f_k \Phi(y, x_k - 2\pi l)} \\
      \phifar(y) &= \sum_{l \notin \mathcal{N}} \sum_{k=0}^{N-1} {f_k \Phi(y, x_k - 2\pi l)}
    \end{align*}
    (i.e. $\phi(y) = \phinear(y) + \phifar(y)$)
  \end{itemize}
\end{frame}

\begin{frame}{Computing the Periodic Summation}{Reworking $\phifar$}
  \begin{itemize}
  \item Substituting a $p$-truncated regular factorization of $\Phi$
    and doing some algebra gives us:
    \begin{align*}
      \phifar(y) = \varepsilon_p(y) + \sum_{m=0}^{p-1} c_m R_m(y - x_*)
    \end{align*}
    with $c_m$ given by:
    \begin{align*}
      c_m = \sum_{l \notin \mathcal{N}} \sum_{k=0}^{N-1} f_k a_m(x_k + 2\pi l, x_*)
    \end{align*}
  \end{itemize}
\end{frame}

\begin{frame}{Computing the Periodic Summation}{Least Squares Collocation}
  \begin{itemize}
  \item We have $\phi = \phinear + \phifar$.
    \pause
  \item For an integer $l$, we also have $\phi(y) = \phi(y + 2\pi l)$.
    \pause
  \item This implies:
    \begin{align*}
      \phifar(y + 2\pi l) - \phifar(y) = \phinear(y) - \phinear(y + 2\pi l)
    \end{align*}
  \end{itemize}
\end{frame}

\begin{frame}{Computing the Periodic Summation}{Least Squares Collocation}
  \begin{itemize}
  \item Substituting our expression for $\phifar$
    in terms of $c_m$'s, we get:
    \begin{align*}
      \phifar(y + 2\pi l) \;-\; &\phifar(y) = \varepsilon_p(y + 2\pi l) - \varepsilon_p(y) \; + \\
      &\sum_{m=0}^{p-1} c_m \parens{R_m(y + 2\pi l - x_*) - R_m(y - x_*)}
    \end{align*}
  % \item We can clean this up a little using the $2\pi l$-forward difference operator $\Delta_{2\pi l}[f](y) = f(y + 2\pi l) - f(y)$:
  %   \begin{align*}
  %     -\Delta_{2\pi l}[\phinear](y) = \Delta_{2\pi l}[\varepsilon_p](y) + \sum_{m=0}^{p-1} c_m \Delta_{2\pi l}[R_m](y - x_*)
  %   \end{align*}
  \end{itemize}
\end{frame}

\begin{frame}{Computing the Periodic Summation}{Least Squares Collocation}
  \begin{itemize}
  \item Consider a set of \textbf{check points} $Z = \set{z_l}$ so that:
    \begin{align*}
      \pi < |z_l - x_*| < r
    \end{align*}
  \item For each of these points, we can pick a $k$
    so that $z_l + 2\pi k$ satisfies:
    \begin{align*}
      |z_l + 2\pi k - x_*| < \pi
    \end{align*}
  \end{itemize}
\end{frame}

\begin{frame}{Computing the Periodic Summation}{Least Squares Collocation}
  \begin{itemize}
  \item If we consider $L > p$
    of these points, we can form $L$
    \textbf{collocation equations} using the preceding equation.
    \pause
  \item If we stack these equations, we get:
    \begin{align*}
      -\bold{R} \bold{c} = \boldsymbol{\phi}_{\operatorname{near}} + \boldsymbol{\varepsilon}_p
    \end{align*}
    \pause
  \item We can solve for $\bold{c}$ using least squares methods.
  \item Choose $L \gg p$ to reduce the error.
  \end{itemize}
\end{frame}

\begin{frame}{Computing the Periodic Summation}{The Algorithm}
  \begin{itemize}
  \item Schematically, the periodic summation algorithm is as follows:
    \begin{enumerate}
    \item Pick parameters $r, R, n,$ and $p$.
    \item Choose a set of check points and find their periodic offsets.
    \item Extend $x_k$'s to periodically tile $[-2\pi n, 2\pi(n+1))$.
    \item Use the FMM to evaluate $\phinear$
      at the $y_j$'s and the checkpoint pairs.
    \item Solve the collocation equations for $\bold{c}$.
    \item Compute $\phifar$ at the $y_j$'s using $\bold{c}$.
    \item Evaluate $\phi = \phinear + \phifar$ at the $y_j$'s.
    \end{enumerate}
  \end{itemize}
\end{frame}

\begin{frame}{Check Points}
  \begin{itemize}
  \item The check points can be chosen in different ways, how can we choose good ones?
    \pause
  \item Heuristically, for now... Some distributions to try:
    \begin{enumerate}
    \item Linearly spaced.
    \item Uniformly distributed.
    \item Gaussian quadrature abscissae.
    \item Kumaraswamy and/or beta distribution.
    \end{enumerate}
  \end{itemize}
\end{frame}

\begin{frame}{Complex Check Points}
  \begin{itemize}
  \item Professor Gumerov suggested solving the problem in the
    complex domain with complex check points.
    \begin{figure}[h]
      \centering
      \vspace{0.75cm}
      \begin{tikzpicture}
        \draw[<-] (-3, 0) -- (-1.1, 0);

        \fill[black] (-0.7, 0) circle (0.5pt);
        \fill[black] (-0.8, 0) circle (0.5pt);
        \fill[black] (-0.9, 0) circle (0.5pt);

        \draw[-] (-0.5, 0) -- (2.5, 0);
        \fill[black] (1, 0) circle (1pt);
        \draw[-] (1, 0.1) -- (1, -0.1) node[below] {$x_\star$};
        
        \fill[black] (2.7, 0) circle (0.5pt);
        \fill[black] (2.8, 0) circle (0.5pt);
        \fill[black] (2.9, 0) circle (0.5pt);

        \draw[->] (3.1, 0) -- (5, 0);

        \draw[-] (0, 0.1) -- (0, -0.1) node[below] {$0$};
        \draw[-] (2, 0.1) -- (2, -0.1) node[below] {$2\pi$};

        \draw[-] (-2, 0.1) -- (-2, -0.1) node[below] {$\pi - r$};
        \draw[-] (4, 0.1) -- (4, -0.1) node[below] {$\pi + r$};

        \draw[dashed] (1, 0) circle (2.25);
        \draw[dashed] (1, 0) circle (1);

        \fill[black] (2.125, 1.9485571585149868) circle (1pt);
        \fill[black] (0.125, 0.05144284148501321) circle (1pt);
        \draw[->] (2.125, 1.9485571585149868) -- (0.125, 0.05144284148501321);

        \fill[black] (1.1624359867364369, 2.244128906772728) circle (1pt);
        \fill[black] (1.1624359867364369, 0.244128906772728) circle (1pt);
        \draw[->] (1.1624359867364369, 2.244128906772728) -- (1.1624359867364369, 0.244128906772728);

        \fill[black] (-0.5909902576697323, -1.590990257669732) circle (1pt);
        \fill[black] (0.40900974233026766, -0.5909902576697319) circle (1pt);
        \draw[->] (-0.5909902576697323, -1.590990257669732) -- (0.40900974233026766, -0.5909902576697319);

      \end{tikzpicture}
    \end{figure}
  \end{itemize}
\end{frame}

\begin{frame}{Progress}
  \begin{itemize}
  \item I've implemented a 1D MLFMM for this kernel in Julia.
    \pause
  \item I've implemented the periodic summation as described and am debugging it...
    \pause
  \item \textbf{Todo} (in order of priority):
    \begin{enumerate}
    \item Vet check point distributions numerically.
      \pause
    \item Numerically compare the final nonequispaced IFFT algorithm with
      current fastest implementation.
      \pause
    \item Complexity analysis.
      \pause
    \item Profile and optimize overall algorithm (1D MLFMM and
      periodic summation).
    \end{enumerate}
  \end{itemize}
\end{frame}

\begin{frame}{Future Work}
  \begin{itemize}
  \item \textbf{Forward interpolation algorithm!}
    \pause
    \begin{enumerate}
    \item Try and fit the corresponding interpolation equations into this framework.
    \item Implement the necessary 1D MLFMM.
    \end{enumerate}
    \pause
  \item Analysis of check points.
    \pause
  \item Implement complex version of ``1D'' MLFMM (should be easy).
    \pause
    \begin{enumerate}
    \item Could be used for the algorithm in 2D?
    \item Try complex check points.
    \end{enumerate}
    \pause
  \item Thoroughly optimize algorithm (parallelization on CPU and GPU, adaptive FMM?) in Julia.
    \pause
  \item Rewrite kernel of algorithm in C, expose to Julia and other languages?
    \pause
  \item Look at this algorithm in dimensions 3 and greater...
  \end{itemize}
\end{frame}

\end{document}