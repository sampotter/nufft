\section{Numerical Results}

\emph{Because of finals week, I wasn't able to conduct quite as many
  tests as I would have liked. I verified that my implementation of
  the MLFMM exhibited the expected asymptotic behavior by comparing it
  with direct evaluation $\phi$ (see
  Figure~\ref{fig:direct-vs-mlfmm}). I also compared to speed of
  computing the inverse interpolation using the matrix product
  directly with computing it using the periodic summation method.}

\emph{The MLFMM exhibited the expected speedup. The speedup wasn't
  nearly as good as I would have liked, but the implementation was
  relatively straightforward and unoptimized. There is a lot of
  low-hanging fruit waiting\----see the section on optimizing the
  MLFMM.}

\emph{The periodic summation method performed much better than
  directly using the matrix product, but a word of warning: the
  timings for the matrix product include both the matrix
  multiplication and the computation of the matrix. The computation of
  the matrix accounted for almost the entirety of the running time,
  and is maybe not a fair comparison. The ``ground truth'' method
  could be appropriately sped up if it were itself
  optimized\----what's more, other forms of the product could be
  explored. For instance, consider any step of the derivation of the
  interpolation formula. In some sense, this plot was included mostly
  as a stand-in for a better comparison, ideally to be made against,
  e.g., Potts's NFFT library. In any case, the fact that the
  asymptotically complex computation of the interpolation matrix is
  avoided by this algorithm is a good thing, and is on display here.}

\begin{figure} 
  \centering
  \caption{direct computation of $\phi$ versus evaluation with our
    MLFMM implemented in Julia.}
  \includegraphics{fmm_speed.eps}
  \label{fig:direct-vs-mlfmm}
\end{figure}

\begin{figure}
    \centering
    \caption{interpolation using the ground truth matrix product and
      using the periodic summation method and
      Equation~\ref{eqn:gj-per-sum-form}.}
    \includegraphics{interp_speed.eps}
\end{figure}

The next things that will be examined by way of numerical
experimentation are:
\begin{itemize}
\item The effect of the check point distribution on the error.
\item The relative of error of interpolation of different bandlimited
  functions (our different test Fourier series).
\item How our method compares to the Potts NFFT (and to any other
  NUFFT implementations?).
\end{itemize}
%%% Local Variables:
%%% mode: latex
%%% TeX-master: "writeup.tex"
%%% End:
