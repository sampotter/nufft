\section{Numerical Results}

\begin{itemize}
\item \TODO\ include figure of approximated polynomial, possibly with
  ``off'' polynomials that are computed from checkpoints with higher
  condition number.
\item \TODO\ compare manually computing $c_0$ with our method of
  estimating $c_0$.
  \begin{itemize}
  \item \TODO\ include the plot which shows how our method for
    estimating $c_0$ diverges as $\delta \downarrow 0$.
  \item \TODO\ include a plot which accurately shows how our estimated
    $c_0$ is quite accurate as compared to manually computing $c_0$
    terms. Have the plot include several different choices of $\delta$
    (i.e. $\delta = 10^{-1}, 10^{-2}, 10^{-3}, \hdots$), use
    $L = 10^1, 10^2, 10^3, 10^4, \hdots$---i.e.\ logarithmic on both
    scales.
  \end{itemize}
\item \TODO\ compare our Vandermonde inverse approach with using
  something like the SVD.\@
\end{itemize}

% \begin{figure}[h]
%   \centering
%   \begin{tikzpicture}
  \begin{loglogaxis}[
    title={$\ell_2$ Error},
    xlabel={$K$},
    ylabel={Error},
    legend entries={Greengard, IRT, NUFFT, NFFT},
    legend style={
      at={(1.03, 0.5)},
      anchor=west
    }]
    \addplot table {../../py/new/data_l2_error_greengard.dat};
    \addplot table {../../py/new/data_l2_error_irt.dat};
    \addplot table {../../py/new/data_l2_error_nufft.dat};
    \addplot table {../../py/new/data_l2_error_potts.dat};
  \end{loglogaxis}
\end{tikzpicture}
%   \caption{test}\label{fig:l2-error}
% \end{figure}

\begin{figure}[h]
  \centering
  \begin{tikzpicture}
  \begin{loglogaxis}[
    thick,
    title=Triangle,
    ymin=1e-8,
    ymax=1e0]
    % \addplot table {../../py/new/data_linf_error_greengard_tri.dat};
    \addplot table {../../py/new/data_linf_error_irt_tri.dat};
    \addplot table {../../py/new/data_linf_error_potts_tri.dat};
    \addplot table {../../py/new/data_linf_error_nufft_tri.dat};
  \end{loglogaxis}
\end{tikzpicture}
\begin{tikzpicture}
  \begin{loglogaxis}[
    thick,
    title=Square,
    ymin=1e-8,
    ymax=1e0]
    % \addplot table {../../py/new/data_linf_error_greengard_square.dat};
    \addplot table {../../py/new/data_linf_error_irt_square.dat};
    \addplot table {../../py/new/data_linf_error_potts_square.dat};
    \addplot table {../../py/new/data_linf_error_nufft_square.dat};
  \end{loglogaxis}
\end{tikzpicture}
\begin{tikzpicture}
  \begin{loglogaxis}[
    thick,
    title=Semicircle,
    ymin=1e-8,
    ymax=1e0,
    legend entries={IRT, NFFT, Algorithm 2},
    legend style={
      legend pos=south west
    }]
    % \addplot table {../../py/new/data_linf_error_greengard_semi.dat};
    \addplot table {../../py/new/data_linf_error_irt_semi.dat};
    \addplot table {../../py/new/data_linf_error_potts_semi.dat};
    \addplot table {../../py/new/data_linf_error_nufft_semi.dat};
  \end{loglogaxis}
\end{tikzpicture}
\begin{tikzpicture}
  \begin{loglogaxis}[
    thick,
    title=Sawtooth,
    ymin=1e-8,
    ymax=1e0]
    % \addplot table {../../py/new/data_linf_error_greengard_saw.dat};
    \addplot table {../../py/new/data_linf_error_irt_saw.dat};
    \addplot table {../../py/new/data_linf_error_potts_saw.dat};
    \addplot table {../../py/new/data_linf_error_nufft_saw.dat};
  \end{loglogaxis}
\end{tikzpicture}

  \caption{test}\label{fig:linf-error}
\end{figure}

\begin{figure}[h]
  \centering
  \begin{tikzpicture}
  \begin{loglogaxis}[
    thick,
    xlabel={$K$},
    ylabel={Time (s.)},
    legend entries={Greengard, IRT, NUFFT, NFFT, IFFT},
    legend style={
      at={(1.03, 0.5)},
      anchor=west
    }]
    \addplot table {../../py/new/data_timings_greengard.dat};
    \addplot table {../../py/new/data_timings_irt.dat};
    \addplot table {../../py/new/data_timings_nufft.dat};
    \addplot table {../../py/new/data_timings_potts.dat};
    \addplot table {../../py/new/data_timings_ifft.dat};
  \end{loglogaxis}
\end{tikzpicture}

  \caption{test}\label{fig:timings}
\end{figure}

\begin{figure}[h]
  \centering
  \begin{tikzpicture}
  \begin{axis}[
    xmode=log,
    zmode=log,
    view/az=-30,
    xlabel={$K$},
    ylabel={$L$},
    zlabel={Time (s)},
    legend entries={NUFFT Runtime, Optimum},
    legend style={
      at={(1.03, 0.5)},
      anchor=west
    }]
    \addplot3[surf] file {../../py/new/data_optL.dat};
    \addplot3+[only marks] file {../../py/new/data_optL_argmins.dat};
  \end{axis}
\end{tikzpicture}

  \caption{test}\label{fig:optL}
\end{figure}

\begin{figure}[h]
  \centering
  \begin{tikzpicture}
  \begin{axis}[
    thick,
    title=Triangle,
    xmin=0,
    xmax=6.28318530718,
    ymin=1e-8,
    ymax=1e0,
    ymode=log]
    % \addplot+[mark=none] table {../../py/new/data_radian_diff_greengard_tri.dat};
    \addplot+[mark=none] table {../../py/new/data_radian_diff_irt_tri.dat};
    \addplot+[mark=none] table {../../py/new/data_radian_diff_nufft_tri.dat};
    \addplot+[mark=none] table {../../py/new/data_radian_diff_potts_tri.dat};
  \end{axis}
\end{tikzpicture}
\begin{tikzpicture}
  \begin{axis}[
    thick,
    title=Square,
    xmin=0,
    xmax=6.28318530718,
    ymin=1e-8,
    ymax=1e0,
    ymode=log]
    % \addplot+[mark=none] table {../../py/new/data_radian_diff_greengard_square.dat};
    \addplot+[mark=none] table {../../py/new/data_radian_diff_irt_square.dat};
    \addplot+[mark=none] table {../../py/new/data_radian_diff_nufft_square.dat};
    \addplot+[mark=none] table {../../py/new/data_radian_diff_potts_square.dat};
  \end{axis}
\end{tikzpicture}
\begin{tikzpicture}
  \begin{axis}[
    thick,
    title=Semicircle,
    xmin=0,
    xmax=6.28318530718,
    ymin=1e-8,
    ymax=1e0,
    ymode=log]
    % \addplot+[mark=none] table {../../py/new/data_radian_diff_greengard_semi.dat};
    \addplot+[mark=none] table {../../py/new/data_radian_diff_irt_semi.dat};
    \addplot+[mark=none] table {../../py/new/data_radian_diff_nufft_semi.dat};
    \addplot+[mark=none] table {../../py/new/data_radian_diff_potts_semi.dat};
  \end{axis}
\end{tikzpicture}
\begin{tikzpicture}
  \begin{axis}[
    thick,
    title=Sawtooth,
    xmin=0,
    xmax=6.28318530718,
    ymin=1e-8,
    ymax=1e0,
    ymode=log,
    legend entries={IRT, NUFFT, NFFT},
    legend style={
        legend pos=south east,
    }]
    % \addplot+[mark=none] table {../../py/new/data_radian_diff_greengard_saw.dat};
    \addplot+[mark=none] table {../../py/new/data_radian_diff_irt_saw.dat};
    \addplot+[mark=none] table {../../py/new/data_radian_diff_nufft_saw.dat};
    \addplot+[mark=none] table {../../py/new/data_radian_diff_potts_saw.dat};
  \end{axis}
\end{tikzpicture}

  \caption{test}\label{fig:raddiff}
\end{figure}

% Local Variables:
% TeX-master: "../paper.tex"
% indent-tabs-mode: nil
% End:
