\section{Future Work}

In this work, we extend and further develop an existing method for
computing the nonuniform FFT, as originally analyzed
in~\cite{Dutt95fastfourierII}. In the original analysis, the method is
mostly passed over in favor of the algorithms derived
in~\cite{dutt-rokhlin-nufft-I}. As demonstrated by the highly
successful NUFFT library, those methods are indeed useful and
powerful. Our work applies results derived in~\cite{periodic-sums} to
make the method presented in~\cite{Dutt95fastfourierII} significantly
more precise, for negligible added runtime cost (in asymptotic terms,
but also in terms of a real implementation). The work establishes the
viability of the FMM as applied to the nonuniform FFT, laying the
ground for future developments. In particular, as the core of the
algorithm makes use of a generic 1D FMM, any improvements to such an
FMM could profitably be exploited by the NUFFT.\@ Such improvements
may be application dependent.

Due to the low communication costs of the FMM, it is well-suited to
parallel implementation, and parallel CPU, GPU~\cite{gumerov2008fast},
heterogeneous CPU/GPU~\cite{qi-hu-thesis}, and distributed
implementations exist and have been investigated to varying
degrees. Our initial implementation was single-threaded and
comparisons were done with other libraries that were compiled to run
in a single thread.

With respect to parallel implementations of the FMM, most development
has been around ``uniform'' FMMs---those whose underlying translation
trees are computed to a uniform depth. Research has gone into
so-called adaptive FMMs (e.g.~\cite{fmm-helmholtz, adaptive-fmm}), but
parallel adaptive FMMs have not been research as intensively, due to
the recursive nature of the adaptive FMM.\@ However, as many
applications of the nonuniform FFT involve fixed sets of target
points, precomputation of the translation trees of adaptive FMMs may
prove to be a good match for the parallel nonuniform FFT.\@ Some
examples of such applications include non-Cartesian MRI, which employs
nonuniform grids in order to speed scanning patterns. Other such
applications exist---in audio signal processing, the constant-$Q$
transform requires a fixed, logarithmic grid of target points for a
given frequency ratio~\cite{constant-q}.

Finally, the strategy of our implementation is to take advantage of
the well-developed and existing research surrounding the FMM.\@ Common
kernels have been researched intensively, and the Cauchy kernel is no
exception. While our implementation takes advantage of the
straightforward (but perhaps somewhat na\"{\i}ve) $R$-expansions and
$S$-expansions that come by way of the Taylor series expansion of the
Cauchy kernel, other methods will be investigated---for instance, a
Chebyshev expansion-based approach is presented in~\cite{dutt1996fast}
which has tighter error bounds.


% Local Variables:
% TeX-master: "../paper.tex"
% indent-tabs-mode: nil
% End:
