\section{Cauchy Kernel $S$ and $R$ Factorizations}

We briefly derive the $S$
and $R$
factorizations for the Cauchy kernel $\Phi(y, x) = \parens{y - x}^{-1}$.
For each factorization, we consider a fixed expansion center
$x_* \in \mathbb{R}$
and two general arguments $x, y \in \mathbb{R}$\----the source point and evaluation point, respectively.  Then, for
the regular expansion, we can write:
\begin{align*}
  \Phi(y, x) = {1 \over y - x} = -{1 \over x - x_* - (y - x_*)} = -{1 \over x - x_*} \sum_{m=0}^\infty \parens{y - x_* \over x - x_*}^m,
\end{align*}
requiring $\abs{y - x_*} < \abs{x - x_*}$.
This suggests that we define $R_m(y - x_*) = \parens{y - x_*}^m$ and $a_m(x, x_*) = -\parens{x - x_*}^{-m-1}$ so that:
\begin{align*}
  \Phi(y, x) = \sum_{m=0}^\infty a_m(x, x_*) R_m(y - x_*).
\end{align*}
For the singular kernel, following a similar derivation, we define $S_m(y - x_*) = \parens{y - x_*}^{-m-1}$ and $b_m(x, x_*) = \parens{x - x_*}^m$, giving us:
\begin{align*}
  \Phi(y, x) = \sum_{m=0}^\infty b_m(x, x_*) S_m(y - x_*),
\end{align*}
where we require that $|x - x_*| < |y - x_*|$. \\

\subsection{Orthogonalized $R$ Factorization}

In this section we orthogonalize the set of functions
$\set{R_m(y-\pi)}_{m=0}^\infty$. First, for $n \geq 1$ and
$i, j \in \set{0, 1, 2, \hdots}$, we compute the inner product between
$R_i(y - \pi)$ and $R_j(y - \pi)$ in $L^2[-2\pi n, 2\pi(n+1)]$:
\begin{align*}
  \ip{R_i(y - \pi), R_j(y - \pi)} &= \int_{-2\pi n}^{2\pi(n+1)} R_i(y - \pi)R_j(y - \pi) dy \\
  &= \int_{-2\pi n}^{2\pi(n+1)} {(y - \pi)}^{i+j} dy \\
  &= \left. \frac{1}{i+j+1} (y - \pi)^{i+j+1} \right|_{y = -2\pi n}^{2\pi(n+1)} \\
  &= \frac{\pi^{i+j+1} (2n + 1)^{i+j+1}}{i+j+1} \cdot \left\{ \begin{tabular}{cl} 2 & if $i + j$ is even, \\ 0 & otherwise. \end{tabular} \right.
\end{align*}
From this, we readily compute that
$\norm{R_0(y - \pi)}_{L^2} = \sqrt{2\pi(2n + 1)}$. We apply
Gram-Schmidt to orthogonalize $\set{R_m(y - \pi)}_{m=0}^\infty$,
generating a new function sequence
$\set{\tilde{R}_m(y - \pi)}_{m=0}^\infty$. This involves setting:
\begin{align*}
  \tilde{R}_0(y - \pi) \equiv \frac{R_0(y - \pi)}{\norm{R_0(y - \pi)}_{L^2}} = \frac{1}{\sqrt{2\pi(2n + 1)}}.
\end{align*}
The latter terms in the s

%%% Local Variables:
%%% mode: latex
%%% TeX-master: "writeup.tex"
%%% End:
