\section{Introduction}

The discrete Fourier transform (DFT) is an orthogonal linear
transformation which is a discrete analog of the continuous Fourier
transform. Many problems in science and engineering yield to the
methods of Fourier analysis, so much so that when the fast Fourier
transform (FFT) algorithm of Cooley and Tukey was disclosed in 1965,
much of the focus of research in the field of signal processing
shifted from research in analog signal processing to digital signal
processing, owing to the newfound feasibility of previously
speculative numerical methods~\cite{book:dsp75}. Of course, this
algorithm has gone on to enable the efficient solution of a myriad of
problems since.

A limitation of the real FFT is that its domain consists of real
vectors whose components are assumed to correspond to equally spaced
samples of some underlying continuous signal; likewise, the range is
comprised of coefficients of equally spaced frequency components. When
presented with a nonuniformly sampled input signal or when one's goal
is to determine the strength of frequency components which are
unequally spaced, the usual FFT is of no use without interpolating
before or after the transform. A variety of methods have been
introduced to formalize an approach to computing such a nonuniform DFT
or FFT.\@

In this work, we extend a method for computing the nonuniform FFT
which makes use of the fast multipole method
(FMM)~\cite{Dutt95fastfourierII}. In particular, making use of recent
work inspired by the kernel independent FMM, we append a step to this
method which takes a negligible amount of time but increases the
accuracy of the result. Algorithms are derived and their complexity is
analyzed, relevant error bounds are established, and numerical results
are presented.

% Local Variables:
% TeX-master: "../paper.tex"
% indent-tabs-mode: nil
% End:
