\section{Test Series}

A leading statistician, David Donoho, in an audaciously titled but
basically pragmatic online article discusses how to be a highly cited
researcher in the mathematical
sciences~\cite{online:how-to-be-highly-cited}. The article references
a collection of papers that he and several of his collaborators
published on wavelets. Donoho points out that the inclusion of
synthetic test cases with reproducible computational research can go a
long way towards generating further research. In his case, papers were
published which involved, e.g., tuning the parameters of previously
published algorithms in order to improve the performance of the
algorithms on test cases published with them. Donoho questions the
value of this research, but the point is made.

\begin{figure}[h]
  \centering
  \includegraphics{semicircle.eps} \includegraphics{triangle.eps} \\
  \includegraphics{sawtooth.eps} \includegraphics{square.eps} \\
  \caption{plots of the first several truncated Fourier series of our
    test functions. The domain of each plot is $x \in [0, 2\pi]$.}
  \label{fig:test-series}
\end{figure}

At the onset of this project, we followed this advice and selected
four test series to test our algorithm on, as well as to verify the
validity of ground truth methods and intermediate steps in deriving
the final algorithm. This was exceedingly helpful and led to the
discovery of several errors. Our view is that the utility of such test
cases is two-fold: on the one hand, they may provide an intuitive
ground truth that can be used to verify the correctness of the
algorithm. More importantly, they might be able to shed light on the
sorts of qualitative behavior the algorithm might exhibit. For
instance, Fourier series exhibit slow convergence at discontinuities,
owing to the Gibbs phenomenon. We selected test cases that intended to
highlight this effect.

We made use of four test cases while developing our algorithm: the
square wave~\cite{online:mathworld-fourier-series-square}, the saw
wave~\cite{online:mathworld-fourier-series-sawtooth}, the triangle
wave~\cite{online:mathworld-fourier-series-triangle}, and the
$2\pi$-periodic
semicircle~\cite{online:mathworld-fourier-series-semicircle}. Plots of
the first few truncated Fourier series of each are included (see
Figure~\ref{fig:test-series}). We were not able to get so far with
this project that we were able to suss out the qualititative
differences in the behavior of the algorithm across our different test
functions\----but it will be pursued in the future, for the reasons
outlined earlier.
