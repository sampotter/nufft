\section{Numerical Results}

To evaluate our method, we chose several real-valued bandlimited test
functions to use in comparisons made with existing software libraries
that can be used to solve the same problem. We chose four, well-known
and canonical trigonometric series: the triangle, square, semicircle,
and saw waves. For a given bandlimit $\bandlimit{}$, approximations to
each of these are given by:
\begin{align}
  \tri(x) &= \idftsum{\unifptindex} e^{i\unifptindex\unifpt{}} \cdot \begin{cases}  0 & \mbox{if } \unifptindex{} \mbox{ mod } 2 \equiv 0 \\ \frac{i{(-1)}^{(\unifptindex{}+1)/2}}{\pi{}^2\unifptindex{}^2} & \mbox{otherwise} \end{cases} \ \label{eq:triangle} \\
  \sqr(x) &= \idftsum{\unifptindex} e^{i\unifptindex\unifpt{}} \cdot \begin{cases} 0 & \mbox{if } \unifptindex{} \mbox{ mod } 2 \equiv 0 \\ \frac{-2i}{\pi{}\unifptindex{}} & \mbox{otherwise} \end{cases} \label{eq:square} \\
  \semi(x) &= \idftsum{\unifptindex} e^{i\unifptindex\unifpt{}} \cdot \begin{cases} \nicefrac{\pi^2}{4} & \mbox{if } \unifptindex{} = 0 \\ \frac{\pi{}{(-1)}^\unifptindex{} J_1(\pi{}\unifptindex{})}{2\unifptindex{}} & \mbox{otherwise} \end{cases} \label{eq:semicircle} \\
  \saw(x) &= \idftsum{\unifptindex} e^{i\unifptindex\unifpt{}} \cdot \begin{cases} 0 & \mbox{if } k = 0 \\ \frac{i{(-1)}^\unifptindex}{\pi\unifptindex} & \mbox{otherwise} \end{cases} \label{eq:sawtooth}
\end{align}
where $J_1$ is a Bessel function of the first kind. Plots of several approximations of these trigonometric series are included for reference (see Figure~\ref{fig:testfunc}).

The software libraries used in comparisons include
NFFT3~\cite{using-nfft3}, NUFFT~\cite{accelerating-nufft,
  type3-nufft-apps}, and IRT~\cite{Fessler:2003dz}. The first library
consists of a large suite of methods which have a similar flavor as to
the nonuniform FFT.\@ The second library is a highly optimized library
based on analysis of the nonuniform FFT initially carried out
in~\cite{dutt-rokhlin-nufft-I}, and refined
in~\cite{accelerating-nufft, type3-nufft-apps} and in other articles
by the same authors. The third library represents a separate approach
based on min-max interpolation.

\begin{figure}[h]
  \centering
  \begin{tikzpicture}
  \pgfplotstableread{../../py/new/data_testfunc_tri.dat}\tritable
  \begin{axis}[
    thick,
    title=Triangle,
    xmin=0,
    xmax=6.28318530718]
    \addplot+[smooth, mark=none] table[x=x, y=K4] {\tritable};
    \addplot+[smooth, mark=none] table[x=x, y=K5] {\tritable};
    \addplot+[smooth, mark=none] table[x=x, y=K6] {\tritable};
    \addplot+[smooth, mark=none] table[x=x, y=K7] {\tritable};
    \addplot+[smooth, mark=none] table[x=x, y=K8] {\tritable};
    \addplot+[smooth, mark=none] table[x=x, y=K9] {\tritable};
    \addplot+[smooth, mark=none] table[x=x, y=K10] {\tritable};
  \end{axis}
\end{tikzpicture}
\begin{tikzpicture}
  \pgfplotstableread{../../py/new/data_testfunc_square.dat}\squaretable
  \begin{axis}[
    thick,
    title=Square,
    xmin=0,
    xmax=6.28318530718,
    legend entries={$K = 4$, $K = 5$, $K = 6$, $K = 7$, $K = 8$,
      $K = 9$, $K = 10$},
    legend style={legend pos=north east}]
    \addplot+[smooth, mark=none] table[x=x, y=K4] {\squaretable};
    \addplot+[smooth, mark=none] table[x=x, y=K5] {\squaretable};
    \addplot+[smooth, mark=none] table[x=x, y=K6] {\squaretable};
    \addplot+[smooth, mark=none] table[x=x, y=K7] {\squaretable};
    \addplot+[smooth, mark=none] table[x=x, y=K8] {\squaretable};
    \addplot+[smooth, mark=none] table[x=x, y=K9] {\squaretable};
    \addplot+[smooth, mark=none] table[x=x, y=K10] {\squaretable};
  \end{axis}
\end{tikzpicture}
\begin{tikzpicture}
  \pgfplotstableread{../../py/new/data_testfunc_semi.dat}\semitable
  \begin{axis}[
    thick,
    title=Semicircle,
    xmin=0,
    xmax=6.28318530718]
    \addplot+[smooth, mark=none] table[x=x, y=K4] {\semitable};
    \addplot+[smooth, mark=none] table[x=x, y=K5] {\semitable};
    \addplot+[smooth, mark=none] table[x=x, y=K6] {\semitable};
    \addplot+[smooth, mark=none] table[x=x, y=K7] {\semitable};
    \addplot+[smooth, mark=none] table[x=x, y=K8] {\semitable};
    \addplot+[smooth, mark=none] table[x=x, y=K9] {\semitable};
    \addplot+[smooth, mark=none] table[x=x, y=K10] {\semitable};
  \end{axis}
\end{tikzpicture}
\begin{tikzpicture}
  \pgfplotstableread{../../py/new/data_testfunc_saw.dat}\sawtable
  \begin{axis}[
    thick,
    title=Sawtooth,
    xmin=0,
    xmax=6.28318530718]
    \addplot+[smooth, mark=none] table[x=x, y=K4] {\sawtable};
    \addplot+[smooth, mark=none] table[x=x, y=K5] {\sawtable};
    \addplot+[smooth, mark=none] table[x=x, y=K6] {\sawtable};
    \addplot+[smooth, mark=none] table[x=x, y=K7] {\sawtable};
    \addplot+[smooth, mark=none] table[x=x, y=K8] {\sawtable};
    \addplot+[smooth, mark=none] table[x=x, y=K9] {\sawtable};
    \addplot+[smooth, mark=none] table[x=x, y=K10] {\sawtable};
  \end{axis}
\end{tikzpicture}

  \caption{test}\label{fig:testfunc}
\end{figure}

Our first numerical test consists of---for a varying choice of
bandlimit $\bandlimit$---evaluating the nonuniform DFT directly from
(\ref{eq:first-interpolation}), using each of the implementation
provided by each library and Algorithm 2 to evaluate the nonuniform
DFT, and plotting the corresponding $\ell_\infty$ error between each
result and the groundtruth result evaluated using
(\ref{eq:first-interpolation}) (see Figure~\ref{fig:linf-error}). The
evaluation points for each experiment were chosen uniformly at random
from the interval $[0, 2\pi)$. The algorithm used in the NUFFT library
involves a fast Gaussian gridding approach, and the interface provides
an error parameter which is gauranteed to be attained (with machine
epsilon being the default). Because of this, we did not include it in
this test.

\begin{figure}[h]
  \centering
  \begin{tikzpicture}
  \begin{loglogaxis}[
    thick,
    title=Triangle,
    ymin=1e-8,
    ymax=1e0]
    % \addplot table {../../py/new/data_linf_error_greengard_tri.dat};
    \addplot table {../../py/new/data_linf_error_irt_tri.dat};
    \addplot table {../../py/new/data_linf_error_potts_tri.dat};
    \addplot table {../../py/new/data_linf_error_nufft_tri.dat};
  \end{loglogaxis}
\end{tikzpicture}
\begin{tikzpicture}
  \begin{loglogaxis}[
    thick,
    title=Square,
    ymin=1e-8,
    ymax=1e0]
    % \addplot table {../../py/new/data_linf_error_greengard_square.dat};
    \addplot table {../../py/new/data_linf_error_irt_square.dat};
    \addplot table {../../py/new/data_linf_error_potts_square.dat};
    \addplot table {../../py/new/data_linf_error_nufft_square.dat};
  \end{loglogaxis}
\end{tikzpicture}
\begin{tikzpicture}
  \begin{loglogaxis}[
    thick,
    title=Semicircle,
    ymin=1e-8,
    ymax=1e0,
    legend entries={IRT, NFFT, Algorithm 2},
    legend style={
      legend pos=south west
    }]
    % \addplot table {../../py/new/data_linf_error_greengard_semi.dat};
    \addplot table {../../py/new/data_linf_error_irt_semi.dat};
    \addplot table {../../py/new/data_linf_error_potts_semi.dat};
    \addplot table {../../py/new/data_linf_error_nufft_semi.dat};
  \end{loglogaxis}
\end{tikzpicture}
\begin{tikzpicture}
  \begin{loglogaxis}[
    thick,
    title=Sawtooth,
    ymin=1e-8,
    ymax=1e0]
    % \addplot table {../../py/new/data_linf_error_greengard_saw.dat};
    \addplot table {../../py/new/data_linf_error_irt_saw.dat};
    \addplot table {../../py/new/data_linf_error_potts_saw.dat};
    \addplot table {../../py/new/data_linf_error_nufft_saw.dat};
  \end{loglogaxis}
\end{tikzpicture}

  \caption{test}\label{fig:linf-error}
\end{figure}

The next test was a comparison of the runtimes of the different
algorithms for the same choice of parameters as in the first
experiment (Figure~\ref{fig:timings}). As a baseline, we included
timings of the inverse FFT, as provided by Python's numpy
library~\cite{van2011numpy}.

\begin{figure}[h]
  \centering
  \begin{tikzpicture}
  \begin{loglogaxis}[
    thick,
    xlabel={$K$},
    ylabel={Time (s.)},
    legend entries={Greengard, IRT, NUFFT, NFFT, IFFT},
    legend style={
      at={(1.03, 0.5)},
      anchor=west
    }]
    \addplot table {../../py/new/data_timings_greengard.dat};
    \addplot table {../../py/new/data_timings_irt.dat};
    \addplot table {../../py/new/data_timings_nufft.dat};
    \addplot table {../../py/new/data_timings_potts.dat};
    \addplot table {../../py/new/data_timings_ifft.dat};
  \end{loglogaxis}
\end{tikzpicture}

  \caption{test}\label{fig:timings}
\end{figure}

Our timing test confirms our basic hypotheses regarding the speed of
the algorithm. As with all FMM-based algorithm, the scaling of the
algorithm is such that beyond a certain point, the algorithm ``breaks
even'' and is beaten by a direct method. However, for problem sizes
below this threshold, we see substantial speedups.

The error plots are more intriguing. Although Algorithm 2 cannot
compete with the implementation provided by NUFFT without appealing to
a high truncation number $\truncnum$ and neighborhood radius
$\neighborhoodradius$, for reasonably small values, the algorithm
attains the same performance as the highly-tuned NFFT and soundly
outperforms what is provided by IRT.\@ Of particular importance, we
can see that for our choice of test functions, IRT and NUFFT exhibit
different accuracy regimes. In particular, we can see that for $\tri$
and $\sqr$, both Algorithm 2 and NFFT are significantly more accurate
than IRT, and that for some problem sizes NFFT attains machine
precision, and for others NFFT and Algorithm 2 are nearly as accurate
as one another. However, $\semi$ and $\saw$ provide exceptions to this
rule: in the $\semi$ test, NFFT and IRT perform equally poorly, with
Algorithm 2 providing much more accurate results; on the other hand,
the $\saw$ test sees NFFT computing results exactly as accurate as
NFFT.\@ Overall, throughout all of the tests, Algorithm 2 does not
surprising provide machine precision, as NFFT occasionally seems to,
but does provide consistently more accurate results than either NFFT
or IRT.\@

\begin{figure}[h]
  \centering
  \begin{tikzpicture}
  \begin{axis}[
    thick,
    title=Triangle,
    xmin=0,
    xmax=6.28318530718,
    ymin=1e-8,
    ymax=1e0,
    ymode=log]
    % \addplot+[mark=none] table {../../py/new/data_radian_diff_greengard_tri.dat};
    \addplot+[mark=none] table {../../py/new/data_radian_diff_irt_tri.dat};
    \addplot+[mark=none] table {../../py/new/data_radian_diff_nufft_tri.dat};
    \addplot+[mark=none] table {../../py/new/data_radian_diff_potts_tri.dat};
  \end{axis}
\end{tikzpicture}
\begin{tikzpicture}
  \begin{axis}[
    thick,
    title=Square,
    xmin=0,
    xmax=6.28318530718,
    ymin=1e-8,
    ymax=1e0,
    ymode=log]
    % \addplot+[mark=none] table {../../py/new/data_radian_diff_greengard_square.dat};
    \addplot+[mark=none] table {../../py/new/data_radian_diff_irt_square.dat};
    \addplot+[mark=none] table {../../py/new/data_radian_diff_nufft_square.dat};
    \addplot+[mark=none] table {../../py/new/data_radian_diff_potts_square.dat};
  \end{axis}
\end{tikzpicture}
\begin{tikzpicture}
  \begin{axis}[
    thick,
    title=Semicircle,
    xmin=0,
    xmax=6.28318530718,
    ymin=1e-8,
    ymax=1e0,
    ymode=log]
    % \addplot+[mark=none] table {../../py/new/data_radian_diff_greengard_semi.dat};
    \addplot+[mark=none] table {../../py/new/data_radian_diff_irt_semi.dat};
    \addplot+[mark=none] table {../../py/new/data_radian_diff_nufft_semi.dat};
    \addplot+[mark=none] table {../../py/new/data_radian_diff_potts_semi.dat};
  \end{axis}
\end{tikzpicture}
\begin{tikzpicture}
  \begin{axis}[
    thick,
    title=Sawtooth,
    xmin=0,
    xmax=6.28318530718,
    ymin=1e-8,
    ymax=1e0,
    ymode=log,
    legend entries={IRT, NUFFT, NFFT},
    legend style={
        legend pos=south east,
    }]
    % \addplot+[mark=none] table {../../py/new/data_radian_diff_greengard_saw.dat};
    \addplot+[mark=none] table {../../py/new/data_radian_diff_irt_saw.dat};
    \addplot+[mark=none] table {../../py/new/data_radian_diff_nufft_saw.dat};
    \addplot+[mark=none] table {../../py/new/data_radian_diff_potts_saw.dat};
  \end{axis}
\end{tikzpicture}

  \caption{test}\label{fig:raddiff}
\end{figure}

% Local Variables:
% TeX-master: "../paper.tex"
% indent-tabs-mode: nil
% End:
