\section{Complexity}

\subsection{FMM}

In this section, we come up with a rough estimate of the cost of the
MLFMM and an asymptotic estimate of its cost. The FMM used in our
implementation consists of several different parts, which can be
considered independently:
\begin{itemize}
\item The $\SSmat$, $\SRmat$, and $\RRmat$ translations.
\item The construction of $S$ factorizations for the finest boxes.
\item Direct evaluation.
\item Indirect evaluation.
\end{itemize}
We consider each of these items in turn. First, we define the
parameters of the algorithm:
\begin{align*}
L &\equiv \mbox{ the maximum level,} \\
M &\equiv \mbox{ the number of evaluation points,} \\
N &\equiv \mbox{ the number of source points,} \\
p &\equiv \mbox{ the truncation number.}
\end{align*}
We also note that in our estimate of the complexity, we assume that
the source and evaluation points are distributed uniformly.

We let $L$ be the maximum level of the FMM. Then, the hierarchies of
$\SSmat$ and $\RRmat$ translations each correspond to a set of four
perfect binary trees, each of height $L - 1$ (i.e., each tree's root
is a box in the second, or coarsest, level). The number of branches in
a perfect binary tree of height $L - 1$ is $2^{L-1} - 1$. So, the
number of either $\SSmat$ or $\RRmat$ translations is
$4(2^{L-1} - 2)$.

To estimate the number of $\SRmat$ translations, we first recall that
$\RRmat$ translations occur between each box and the boxes in the E4
neighborhood of that box. For the 1D FMM, there are (except at the
boundaries), three such neighbors. Since there are $2^{L-1} - 1$ nodes
in a perfect binary tree of height $L-1$, and since the hierarchy of
boxes consists of four such trees, there are $4(2^{L-1} - 1)$
boxes. Therefore, approximately $12(2^{L-1} - 1)$ $\RRmat$
translations occur.

The $\SSmat$, $\SRmat$, and $\RRmat$ translations are each $O(p^2)$
complexity, where $p$ is the truncation number. Combining this
complexity with our estimates of the number of translations that
occur, we have that the cost due to all translations is:
\begin{align*}
  \underbrace{4(2^{L-1} - 2) O(p^2)}_{\SSmat} + \underbrace{4(2^{L-1} - 2) O(p^2)}_{\SRmat} + \underbrace{12(2^{L-1} - 1) O(p^2)}_{\RRmat} = O(2^Lp^2)
\end{align*}

Next, we consider the construction of $S$-factorizations for the
sources in the finest boxes. Constructing an $S$-factorization has
complexity $O(p)$, and there are $N$ sources, so the overall
complexity is $O(Np)$.

For the direct evaluations, we have that direct evaluation occurs
between every target point and each of the source points in its
neighborhood at level $L$. Since we have assumed that the points are
distributed uniformly, there are approximately
$3 \times 2^{-L} N$ sources in the finest neighborhood of an
evaluation point. Then, since there are $M$ evaluation points, the
complexity of direct evaluation is $O(2^{-L}MN)$. 

As for indirect evaluation, we note that indirect evaluation involves
evaluating an $R$-expansion at each of the target points. Since the
cost of evaluating an $R$-expansion is $O(p)$, the complexity is
$O(Mp)$.

Combining each of these individual parts gives us the following table
of costs:
\begin{center}
  \begin{tabular}{|c|c|}
    \hline
    Translations & $O(2^Lp^2)$ \\
    Forming $S$-Factorizations & $O(Np)$ \\
    Direct Evaluation & $O(2^{-L}MN)$ \\
    Indirect Evaluation & $O(Mp)$ \\
    \hline
    Total & $O(2^{-L}MN + (M + N)p + 2^Lp^2)$ \\
    \hline
  \end{tabular}
\end{center}

\subsection{NUFFT}

The nonuniform (I)FFT algorithm which is studied in this paper
consists primarily of computing the IFFT, followed by an application
of the FMM and the solution of a linear least squares problem (i.e.,
the periodic summation). We detail the required parameters:
\begin{align*}
  K &\equiv \mbox{ the number of grid points,} \\
  J &\equiv \mbox{ the number of interpolation points,} \\
  L &\equiv \mbox{ the maximum level of the FMM,} \\
  p &\equiv \mbox{ the truncation number for the FMM,} \\
  q &\equiv \mbox{ the number of check points ($q \gg p$),} \\
  n &\equiv \mbox{ the radius for the periodic summation.}
\end{align*}
We compute the complexity of the FMM followed by the total cost of the
algorithm.

When the FMM is run as a step of the NUFFT, its source points are the
periodic extension of the grid points, which lie in $[0, 2\pi)$, to
the extended domain $[-2\pi n, 2 \pi (n + 1))$. So, instead of $K$
source points, there are $(2n + 1)K$. The evaluation points consist of
the $J$ interpolation points and $q$ checkpoints and periodic offsets
thereof\----hence, there are $J + 2q$ evaluation points. Choosing
$M = J + 2q$ and $N = (2n + 1)K$, the complexity of the FMM is
$O(2^{-L}(J+q)nK + (J + q + nK)p + 2^Lp^2)$.

The periodic summation step is dominated by the cost of the linear
least squares problem that must be the solved. The complexity for a
general system of $q$ equations in $p$ unknowns is $O(qp^2)$.

All told, the complexity of our NUFFT is:
\begin{align*}
  \underbrace{O(K \log K)}_{\text{IFFT}} + \underbrace{O(2^{-L}(J+q)nK + (J + q + nK)p + 2^Lp^2)}_{\text{FMM}} + \underbrace{O(qp^2)}_{\text{LLS}},
\end{align*}
which can be simplified to $O((2^{-L}(J+q)n + \log K)K + (J + nK)p + (2^L + q)p^2)$.

%%% Local Variables:
%%% mode: latex
%%% TeX-master: "writeup.tex"
%%% End:
