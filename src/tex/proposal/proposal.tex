\documentclass{article}

\usepackage{
  amsmath,
  amssymb,
  cite,
  fullpage
}

\begin{document}

\noindent Sam Potter \\
\noindent AMSC808D \\
\noindent Project Proposal \\

\begin{center}
  \huge{Applying the Periodized FMM to the NFFT}
\end{center}

% \section*{Outline}

% Throughout all this, try and maintain a consistent style of notation.

% \begin{enumerate}
% \item Describe the DFT and explain what problem the FFT solves (citing
%   my Oppenheim \& Schafer book).
% \item Describe the NDFT and what might necessitate its use (citing the
%   Daniel Potts NFFT tutorial?).
% \item Reference the prior art (i.e.\ the Daniel Potts NFFT). Indicate
%   asymptotic complexity of their algorithm clearly.
% \item Explain the FMM.\@
% \item Explain the periodic sum method (citing the periodic sum paper).
% \item Explain our formulation (citing attendant references).
% \item Go over what we will do in the project.
%   \begin{enumerate}
%   \item Come up with a method of generating check points and get some
%     error bounds.
%   \item Implement Julia (or Matlab) package for 1D periodized FMM.\@
%   \item Apply Julia (or Matlab) package to NDFT computation.
%   \item Compare performance with NFFT.\@
%   \end{enumerate}
% \end{enumerate}

% \section*{Draft}

The discrete Fourier transform (DFT) is an orthogonal linear
transformation that serves as a discrete analog of the Fourier
transform. Many problems in science and engineering yield to the
methods of Fourier analysis, so much so that when the fast Fourier
transform (FFT) algorithm of Cooley and Tukey was disclosed in 1965,
much of the focus of research in the field of signal processing
shifted from research in analog signal processing to digital signal
processing, owing to the newfound feasibility of previously
speculative numerical methods~\cite{book:dsp75}. Of course, this
algorithm has gone on to enable the efficient solution of a myriad of
problems since.

A limitation of the $N$-point
FFT is that its domain consists of vectors in $\mathbb{C}^N$
whose components are typically assumed to correspond to equally spaced
samples of some underlying continuous signal; likewise, the range is
comprised of coefficients of equally spaced frequency components. In
the cases where our input signal is sampled unevenly or where we would
like to determine the strength of frequency components which are
unequally spaced, the usual FFT is of no use without interpolating
before or after the transform. A variety of methods have been
introduced to formalize an approach to computing a ``nonequispaced''
discrete Fourier transform making use of
interpolation~\cite{Dutt95fastfourierII,Keiner06nfft3.0}.

In this project, our goal is to explore the practicality of a new
approach to computing the nonuniform IDFT (INDFT) efficiently. Like
these other methods, the approach relies on interpolation, and this
project will specifically consider a fast INDFT (INFFT).

To outline the focus of this project, we consider a function
$f : \mathbb{R} \to \mathbb{C}$
which is $2\pi$-periodic
and band-limited. Then, we define the following sets of points:
\begin{enumerate}
\item $x_k = 2\pi k/N$ for $k = 0, \hdots, N-1$ (\emph{equispaced points}).
\item $f_k = f(x_k)$ (\emph{samples at equispaced points}).
\item $y_j \in [0, 2\pi)$ for $j = 0, \hdots, M - 1$ (\emph{nonequispaced points}).
\item $g_j = f(y_j)$ (\emph{samples at nonequispaced points}).
\item $c_n = {1 \over N} \sum_{k=0}^{N-1} f_k e^{-inx_k}$ (\emph{DFT coefficients of $f_k$'s}).
\end{enumerate}
Schematically, we can consider the following relations:
\begin{align}
  \left\{g_j\right\} \longleftrightarrow \left\{f_k\right\} \longleftrightarrow \left\{c_n\right\}
\end{align}
The first arrow represents an invertible interpolation method and the
second represents a method that computes the DFT and IDFT.\@ Then, a
fast interpolation method combined with the FFT and IFFT could
efficiently compute an answer to the problem of finding the integer
frequency components of the unequally spaced samples.

This project focuses on a method to quickly interpolate the points
$\left\{g_j\right\}$
from $\left\{f_k\right\}$,
resulting in the ``inverse interpolation'' required by the INFFT.\@
That is, our method corresponds to the arrow:
\begin{align*}
  \left\{g_j\right\} \longleftarrow \left\{f_k\right\}.
\end{align*}
In~\cite{Dutt95fastfourierII}, an interpolation method was developed
that makes use of an expansion of the sraightforward formulation in
terms of the following kernel:
\begin{align}
  G(y_j - x_k) = - {1 + i \cot((y_j - x_k)/2) \over 2}.
\end{align}
This method makes the interpolation problem amenable to solution by
the fast multipole method when the kernel is suitably expanded.

In \cite{Gumerov13amethod}, a method for the fast computation of
periodic sums is developed. This method is presented in terms of a
black-box fast summation algorithm, but is ideally matched to the fast
multipole method. As the preceding kernel can be expanded as a
$2\pi$-periodic
sum of Cauchy kernels~\cite{fmmfiltering}, this method for computing
periodic sums can be applied to the interpolation problem under
discussion. A caveat of this method is that it requires a set of
``check points'' which must be distributed carefully in order to
ensure good error bounds. The reference paper develops this in terms
of 3-dimensional periodic sums, but not 1-dimensional periodic sums,
so this would need to be investigated.

Overall, then, and including the distribution of the check points for
the periodic summation, the project will involve working out the
following pieces (not necessarily in this order):
\begin{enumerate}
\item Rederiving the separate interpolation results so that they can
  be placed into a coherent framework.
\item Determination of a method to distribute the check points
  involved in the periodic summation method (and derivation of error
  bounds associated with this distribution).
\item Implementation of the periodic summation method in Julia (or Matlab).
\item Implementation of the 1-dimensional FMM in Julia (or Matlab).
\item Derivation of the asymptotic complexity of this method of
  interpolation and the complexity of the resulting INFFT.\@
\item Numerical comparison of the performance of this method of
  computing the INFFT with existing methods.
\end{enumerate}
Since it is possible that this method might be amenable to being sped
up through parallelization, we are interested in implementing this
project using Julia not only for its general speed when compared to
Matlab but also for how readily code written in it can be
parallelized. Also, Julia has a package manager for fast dissemination
of software and is supposed to have really nice looking plots.

This project addresses the fast computation of the inverse nonuniform
DFT, where the points in the domain are nonuniformly sampled and the
frequency coefficients are computed for uniformly spaced
frequencies. If this work is successful, then it will provide a
starting point for exploring the other combinations of
uniformly/nonuniformly sampled points, as well as the forward
nonuniform DFT. In particular, we note that an expansion exists for
the NFFT such that the periodic summation method could be applied with
the FFM. However, this expansion is more complicated; and the
``forward interpolation'' step itself is also more involved than the
computation of a single matrix-vector product.

\bibliography{proposal}{}
\bibliographystyle{plain}

\end{document}