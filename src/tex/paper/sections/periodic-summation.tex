\section{Periodic Summation}

The FMM sees frequent application in dynamics problems such as the
$n$-body problem. For problems with periodic boundary conditions, the
FMM on its own is not sufficient. Others algorithms are used
instead---for example, Ewald summation, accelerated using the
FFT~\cite{ewald}.\@ Recent work inspired by kernel-independent FMM
algorithms has resulted in a method for computing periodic sums which
is ideally suited for use with the FMM~\cite{periodic-sums}. Our
method makes use of a simplified and accelerated version of this
approach. We briefly derive the necessary results as they pertain to
our problem.

The periodic summation method uses an FMM and least squares fitting to
evaluate a sum which is a sum of weighted kernels which periodically
repeat in some infinite computational domain. In our case, we are
interested in computing the part of (\ref{eq:per-interp-cot-kernel})
which matches this description---namely, we define:
\begin{align}
  \label{eq:phi}
  \phi(\phiarg{}) \defd \sum_{\persumindex{}=-\infty}^\infty \dftsum{\unifptindex{}} \frac{{(-1)}^{\unifptindex{}}\unifval{\unifptindex{}}}{\phiarg{} - \unifpt{\unifptindex{}} - 2\pi{}\persumindex{}},
\end{align}
where we have ingored the constant factor. Next, in order to apply the
method, we select some region of interest, a larger neighborhood which
contains it, and decompose $\phi$ into a term due to the neighborhood
and one due to its complement. Our region of interest is the interval
$[0, 2\pi)$. We fix $\neighborhoodradius\in\N$ and consider it to be
an integer ``neighborhood radius'' and define:
\begin{align}
  \label{eq:neighborhood}
  \persumdomain \defd \set{-n, -n+1, \hdots, n}.
\end{align}
Then, letting:
\begin{align}
  \phinear(\phiarg) &\defd \sum_{\persumindex\in\persumdomain}\dftsum{\unifptindex{}} \frac{{(-1)}^\unifptindex\unifval{\unifptindex}}{\phiarg-\unifpt{\unifptindex}-2\pi\persumindex}, \label{eq:phinear} \\
  \phifar(\phiarg) &\defd \sum_{\persumindex\notin\persumdomain}\dftsum{\unifptindex{}} \frac{{(-1)}^\unifptindex\unifval{\unifptindex}}{\phiarg-\unifpt{\unifptindex}-2\pi\persumindex}, \label{eq:phifar}
\end{align}
we can decompose $\phi$ as:
\begin{align}
  \label{eq:phinear-plus-phifar}
  \phi(\phiarg) = \phinear(\phiarg) + \phifar(\phiarg).
\end{align}

With $\xstar = \pi$, we have that when $\phiarg$ satisfies
$0 \leq \phiarg < 2\pi$, the $R$-factorization:
\begin{align}
  \label{eq:phifar-R-factorization}
  \frac{1}{\phiarg - \unifpt{\unifptindex} - 2\pi\persumindex} = \sum_{\powerindex{}=0}^{\infty}a_m(\unifpt{\unifptindex} + 2\pi\persumindex, \pi)R_m(\phiarg - \pi)
\end{align}
is valid. Making use of (\ref{eq:phifar-R-factorization}), we can
define, for each $m = 0, 1, \hdots$:
\begin{align}
  \label{eq:phifar-over-R-coefs}
  \persumcoef{m} \defd \sum_{\persumindex\notin\persumdomain}\dftsum{\unifptindex{}} {(-1)}^\unifptindex \unifval{\unifptindex} a_m(\unifpt{\unifptindex} + 2\pi\persumindex, \pi),
\end{align}
which allows us to define:
\begin{align}
  \label{eq:phifar-over-R}
  \phifar(\phiarg) = \sum_{\powerindex=0}^\infty \persumcoef{m} R_m(\phiarg - \pi).
\end{align}
Reexpressing $\phifar$ in terms of the $R_m$ basis functions of the
$R$-factorization is the crucial step that enables the method.

The periodic summation method involves using the FMM to compute the
neighborhood term $\phinear$ and then using the FMM and least squares
collocation to compute $\phifar$. In particular, if we choose
$\phiarg \suchthat {-2\pi\neighborhoodradius} \leq \phiarg < 0$ or
$2\pi \leq \phiarg < 2\pi(\neighborhoodradius + 1)$, then there exists
some $\persumindex$ such that $\phiargcp = \phiarg - 2\pi\persumindex$
satisfies $0 \leq \phiargcp < 2\pi$. Since
$\phi(\phiarg) = \phi(\phiargcp)$, we have:
\begin{align}
  \label{eq:phinear-minus-phinear}
  \phinear(\phiarg) - \phinear(\phiargcp) = \phifar(\phiargcp) - \phifar(\phiarg) = \sum_{\powerindex=0}^\infty \persumcoef{m}(R_m(\phiarg - \pi) - R_m(\phiargcp - \pi)).
\end{align}
Fixing $\numcps\in\N$, we let
$\phiarg_0, \hdots, \phiarg_{\numcps - 1}$ and
$\phiargcp_0, \hdots, \phiargcp_{\numcps - 1}$ satisfy the same
properties as $\phiarg$ and $\phiargcp$. Defining the matrices
$\m{\phi} \in \R^L, \m{\persumcoef{}} \in \R^\infty$, and
$\m{R} \in \mathbb{R}^{L\times\infty}$ by:
\begin{align}
  \m{\phi}_\cpindex &= \phinear(\phiarg_\cpindex) - \phinear(\phiargcp_\cpindex), \label{eq:phi-vector} \\
  \m{\persumcoef{}}_m &= \persumcoef{m}, \label{eq:coef-vector} \\
  \m{R}_{l,m} &= R_m(\phiarg_\cpindex - \pi) - R_m(\phiargcp_\cpindex - \pi), \label{eq:fitting-matrix}
\end{align}
the coefficients $\persumcoef{m}$ can be recovered by solving the
corresponding least squares problems, e.g.\ by computing each
$\persumcoef{m}$ from
$\m{\persumcoef{}} \approx \m{R}^\dagger \m{\phi}$.

There are two practical considerations in all of this. First, we only
compute and apply a finite portion of $\m{R}$, although the full
matrix will be used in our error analysis. Second, since
$\m{R}_{l,0} = 0$ for each $l\in\N$, this column can be ignored. As a
result, the question as to what to do about $\persumcoef{0}$
arises---this is addressed in the corresponding section. The resulting
submatrix can be dealt with by making use of a convenient matrix
decomposition, also addressed in a later section.

\begin{figure}[h]
  \centering
  \begin{tikzpicture}
  \begin{axis}[
    zmode=log,
    view/az=150,
    xlabel={$n$},
    ylabel={$p$},
    zlabel={Error}]
    \addplot3[surf] file {../../py/new/data_error_bound.dat};
    \addplot3+[only marks] file {../../py/new/data_error_bound_thresholds1.dat};
    \addplot3+[only marks] file {../../py/new/data_error_bound_thresholds2.dat};
  \end{axis}
\end{tikzpicture}


  \caption{test}\label{fig:errorbound}
\end{figure}

% Local Variables:
% TeX-master: "../paper.tex"
% indent-tabs-mode: nil
% End:
