\section{Periodic Summation (Old)}

This section contains a formulation of the periodic summation method
as applied to our problem. The original presentation focused on a
periodic summation in a space of arbitrary dimension (and, for their
particular application, $\mathbb{R}^3$).  The algorithm presented in
this paper only requires the method to operate in $\mathbb{R}$. In the
original paper, a rectangular honeycomb containing a set of
periodically repeating points is considered, where the sum is to be
evaluated over one cell in the honeycomb. For the evaluation, a
neighborhood of cells is selected, as well as a pair of concentric
spheres. The inner sphere contains the computational domain, and the
outer sphere is chosen so that it has nontrivial intersection with
each cell in the selected neighborhood.

In the single-dimensional case, the corresponding domains are
simplified. Instead of a honeycomb, we work with the intervals
$[2\pi n, 2\pi (n + 1))$, for $n \in \mathbb{Z}$. We think of these
intervals as ``boxes''. Since the function to be evaluated is
$2\pi$-periodic, we can make the following definitions:
\begin{itemize}
\item A number $n \in \nats$, corresponding to the number of boxes
  extending radially from the central box.
\item An index set
  $\mathcal{N} = \set{-n, \hdots, n} \subseteq \nats$.
\item Radii $r, R \in \mathbb{R}$ satisfying
  $\pi < r < R < (2n + 1)\pi$ \emph{(analogs of the concentric
    spheres)}.
\end{itemize}
These definitions are illustrated in Figure~\ref{fig:per-sum-alg}.

The periodic summation method requires a kernel that can be factorized
in a manner appropriate for the FMM. The kernel $\Phi$ factorizes as:
\begin{align*}
  \Phi(y, x) = {1 \over y - x} = \sum_{m=0}^\infty a_m(x, x_*) R_m(y - x_*) = \epsilon^{(p)}(y,x) + \sum_{m=0}^{p-1} a_m(x, x_*) R_m(y - x_*).
\end{align*}
We will denote the $2\pi$-periodic
part of our interpolation formula by $\phi(y)$,
written (ignoring the constant factor):
\begin{align*}
  \phi(y) \equiv \sum_{l=-\infty}^\infty \sum_{k=0}^{2K-1} {f_k \over y - x_k - 2\pi l} = \sum_{l=-\infty}^\infty \sum_{k=0}^{2K-1} f_k \Phi(y, x_k + 2\pi l).
\end{align*}
This is the exact form required for application of the periodic
summation method.

\begin{figure}[h]
  \centering
  \caption{an illustration depicting the domain of the periodic
    summation algorithm as well as relevant parameters.}
  \begin{tikzpicture}
    \draw[-] (-0.6,0) -- (2.6, 0);
    \draw[-] (0, 0.1) -- (0, -0.1) node[below] {$0$};
    \fill[black] (1, 0) circle (1pt);
    \draw[dashed] (1, 0) -- (1, -1) node[below] {$x_* = \pi$};
    \draw[-] (2, 0.1) -- (2, -0.1) node[below] {$2\pi$};
    
    \fill[black] (-0.85, 0) circle (0.5pt);
    \fill[black] (-1.0, 0) circle (0.5pt);
    \fill[black] (-1.15, 0) circle (0.5pt);

    \fill[black] (2.85, 0) circle (0.5pt);
    \fill[black] (3.0, 0) circle (0.5pt);
    \fill[black] (3.15, 0) circle (0.5pt);

    \draw[->] (-1.4, 0) -- (-2.25, 0);
    \draw[-] (-2, 0.1) -- (-2, -0.1) node[below] {$-2\pi n$};

    \draw[->] (3.4, 0) -- (4.25, 0);
    \draw[-] (4, 0.1) -- (4, -0.1) node[below] {$2\pi (n + 1)$};

    \draw[dashed] (1, 1) -- (3.7, 1);
    \draw[-] (1, 1.1) -- (1, 0.9);
    \draw[-] (2.7, 1.3) node[above] {$r + \pi$} -- (2.7, 0.9);
    \draw[-] (3.32, 1.1) -- (3.32, 0.7) node[below] {$R + \pi$};
  \end{tikzpicture}
  \label{fig:per-sum-alg}
\end{figure}

Given our choice of $n$, we can decompose $\phi$ into a near component
and far component. We define $\phinear$ and $\phifar$ by:
\begin{align*}
  \phinear(y) &\equiv \sum_{l \in \mathcal{N}} \sum_{k=0}^{2K-1} f_k \Phi(y, x_k + 2\pi l), \\
  \phifar(y) &\equiv \sum_{l \notin \mathcal{N}} \sum_{k=0}^{2K-1} f_k \Phi(y, x_k + 2\pi l).
\end{align*}
Making use of this decomposition, we would like to build towards
formulating the least squares collocation step of the periodic
summation algorithm. The first step in this direction is reexpressing
$\phifar$ in the correct form. We have that:
\begin{align*}
  \phifar(y)
  &= \sum_{l \notin \mathcal{N}} \sum_{k=0}^{2K-1} f_k \Phi(y, x_k + 2\pi l) \\
  &= \sum_{l \notin \mathcal{N}} \sum_{k=0}^{2K-1} f_k \parens{\epsilon^{(p)}(y, x_k + 2\pi l) + \sum_{m=0}^{p-1} a_m(x_k + 2\pi l, x_*) R_m(y - x_*)} \\
  &= \sum_{l \notin \mathcal{N}} \sum_{k=0}^{2K-1} f_k \epsilon^{(p)}(y, x_k + 2\pi l) + \sum_{m=0}^{p-1} \parens{ \sum_{l \notin \mathcal{N}} \sum_{k=0}^{2K-1} f_k a_m(x_k + 2\pi l, x_*)} R_m(y - x_*).
\end{align*}
This suggests that we define:
\begin{itemize}
\item $\epsilon^{(p)}(y) \equiv \sum_{l \notin \mathcal{N}} \sum_{k=0}^{2K-1} f_k \epsilon^{(p)}(y, x_k + 2\pi l)$,
\item $c_m \equiv \sum_{l \notin \mathcal{N}} \sum_{k=0}^{2K-1} f_k a_m(x_k + 2\pi l, x_*)$.
\end{itemize}
Making use of these definitions, then, we can simplify $\phifar$ by writing:
\begin{align*}
  \phifar(y) = \epsilon^{(p)}(y) + \sum_{m=0}^{p-1} c_m R_m(y - x_*).
\end{align*}

Since $y$ is $2\pi$-periodic, for each $y \in \mathbb{R}$ and
$l \in \mathbb{Z}$, we have that $\phi(y) = \phi(y + 2\pi l)$.
Recalling that $\phi = \phinear + \phifar$, this further implies that
(\emph{N.B.:} this is the negative of the corresponding equation in
the periodic sums paper):
% \begin{align*}
%   \phifar(y) - \phifar(y + 2\pi l) = \phinear(y + 2\pi l) - \phinear(y).
% \end{align*}
\begin{align*}
  \phifar(y + 2\pi l) - \phifar(y) = \phinear(y) - \phinear(y + 2\pi l).
\end{align*}
Then, focusing on $\phifar$, we can write:
\begin{align*}
  \phifar(y + 2\pi l) - \phifar(y) &= \epsilon^{(p)}(y + 2\pi l) - \epsilon^{(p)}(y) + \sum_{m=0}^{p-1} c_m \parens{R_m(y + 2\pi l - x_*) - R_m(y - x_*)} \\
  &= \Delta_{2\pi l}[\epsilon^{(p)}](y) + \sum_{m=0}^{p-1} c_m \Delta_{2\pi l}[R_m](y - x_*),
\end{align*}
where $\Delta_{2\pi l}[\;\cdot\;]$
denotes the $2\pi l$
forward difference operator. Likewise,
$\phinear(y) - \phinear(y + 2\pi l)$ can be written:
\begin{align*}
  \phinear(y) - \phinear(y + 2\pi l) = - \Delta_{2\pi l}[\phinear](y).
\end{align*}

For the collocation step, we assume that we have chosen a set of $L$
\textbf{check points} $Y = \set{y_1, \hdots, y_L}$
that satisfies $\pi < \abs{y_i - \pi} < r$
for each $i = 1, \hdots, L$.
The choice of such a set is nontrivial and is considered in another
section.  Now, if we consider the equation derived in the preceding
paragraph in turn for each of our check points $y_i \in Y$,
we end up with $L$
equations in $p$
unknowns and can correspondingly define
$\bold{R} \in \mathbb{C}^{L \times p}$,
$\bold{c} \in \mathbb{C}^{p \times 1}$,
$\boldsymbol{\phi}_{\operatorname{near}} \in \mathbb{C}^{L \times 1}$,
and $\boldsymbol{\epsilon}^{(p)} \in \mathbb{C}^{L \times 1}$ by:
\begin{itemize}
\item $\parens{\bold{R}}_{i,m} \equiv \Delta_{2 \pi l}[R_m](y_i - x_*)$,
\item $\parens{\bold{c}}_m \equiv c_m$,
\item $\parens{\boldsymbol{\phi}_{\operatorname{near}}}_i \equiv \Delta_{2\pi l}[\phinear](y_i)$,
\item $\parens{\boldsymbol{\epsilon}^{(p)}}_i \equiv \Delta_{2\pi l}[\epsilon^{(p)}](y_i)$.
\end{itemize}
These definitions allow us to write our equations as a matrix equality
as follows:
\begin{align*}
  -\bold{R} \bold{c} = \boldsymbol{\phi}_{\operatorname{near}} + \boldsymbol{\epsilon}^{(p)}.
\end{align*}
Then, the collocation step of the periodic summation algorithm is
tantamount to solving the preceding equation for $\bold{c}$.
We could, for instance, directly use the Moore-Penrose pseudoinverse,
evaluating:
\begin{align*}
  \bold{c} \approx -\bold{R}^\dagger \parens{\boldsymbol{\phi}_{\operatorname{near}} + \boldsymbol{\epsilon}^{(p)}}.
\end{align*}
Of course, there are other least squares methods which can compute
$\bold{c}$ which may also be used.

\subsection*{Check Points}

The original paper describing the periodic summation method makes the
point that the choice of check points used in the algorithm is
nontrivial. The authors explore several different distributions:
\begin{itemize}
\item Points uniformly distributed on a sphere.
\item The zeros of a certain Legendre polynomial.
\item Points which are the solution to the Thomas problem (the
  equilibrium position of electrons contrained to lie on a sphere).
\end{itemize}
The commonality is that each distribution chooses points that lie on a
sphere. In the 1D version of the algorithm, this corresponds to a pair
of points, which allows for a maximum of two check points\----clearly
problematic.

Our goal was to explore the use of distributions that lie in the
region $\pi < |\pi - y| < r$. Several check points distributions were
implemented:
\begin{itemize}
\item Uniformly distributed points.
\item Linearly spaced points.
\item Logarithmically spaced points.
\item Points drawn from a Beta distribution with parameter
  $\alpha = 1$, and $\beta$ to be determined.
\end{itemize}
In the future, we would also like to look into the abscissae of the
Legendre polynomials, along the same lines as in the original
paper. In our experiments, we primarily made use of points drawn from
the Beta distribution, as they had varying density and were simple to
generate. There was not enough time to go into depth with
experimenting with the trade-offs associated with check point
distributions, but this is one of the first things that will be looked
into in future work.

\section{Periodic Summation (New)}

We let $n \in \nats$ and $\mathcal{N} = \set{-n, -n+1, \hdots,
  n}$. Then, for $y$ such that $0 \leq y < 2\pi$, we define:
\begin{align*}
  \phi(y) \equiv \sum_{l=-\infty}^{\infty} \sum_{k=0}^{2K-1} \frac{(-1)^k f_k}{y_j - x_k - 2\pi l},
\end{align*}
so that if we define:
\begin{align*}
  \phinear(y) &\equiv \sum_{l \in \mathcal{N}} \sum_{k=0}^{2K-1} \frac{(-1)^k f_k}{y_j - x_k - 2\pi l}, \\
  \phifar(y) &\equiv \sum_{l \in \mathbb{Z} \backslash \mathcal{N}} \sum_{k=0}^{2K-1} \frac{(-1)^k f_k}{y_j - x_k - 2\pi l},
\end{align*}
we can write $\phi(y) = \phinear(y) + \phifar(y)$.

For $l \in \mathbb{Z} \backslash \mathcal{N}$, we have that a regular
expansion of $(y - x_k - 2\pi)^{-1}$ about $x_* = \pi$ is valid for
$y$ such that $-2\pi n \leq y < 2\pi(n+1)$. Then, for such a choice of
$y$, we can write $\phifar(y)$ as:
\begin{align*}
  \phifar(y) &= \sum_{l \in \mathbb{Z} \backslash \mathcal{N}} \sum_{k=0}^{2K-1} \frac{(-1)^k f_k}{y - x_k - 2\pi l} \\
  &= \sum_{l \in \mathbb{Z} \backslash \mathcal{N}} \sum_{k=0}^{2K-1} (-1)^k f_k \sum_{m=0}^\infty a_m(x_k + 2\pi l, \pi) R_m(y - \pi) \\
  &= \sum_{m=0}^\infty \parens{\sum_{l \in \mathbb{Z} \backslash \mathcal{N}} \sum_{k=0}^{2K-1} (-1)^k f_k a_m(x_k + 2\pi l, \pi)} R_m(y - \pi). & \mathtodo
\end{align*}
From this, we define $c_m$ by:
\begin{align*}
  c_m \equiv \sum_{l \in \mathbb{Z} \backslash \mathcal{N}} \sum_{k=0}^{2K-1} (-1)^k f_k a_m(x_k + 2\pi l, \pi).
\end{align*}

Letting $L \geq 2K$, we choose $\set{y_l}_{l=1}^L$ such that
$0 \leq y_l < 2\pi$. Then, if we choose
$\set{p_l}_{l=1}^L \subseteq \mathcal{N}$ and define
$\tilde{y}_l \equiv y_l + 2\pi p_l$, we have
$\phinear(y_l) + \phifar(y_l) = \phinear(\tilde{y}_l) +
\phifar(\tilde{y}_l)$ by assumption. Then, we can write:
\begin{align*}
  \phinear(\tilde{y}_l) - \phinear(y_l) &= \phifar(y_l) - \phifar(\tilde{y}_l) = \sum_{m=0}^\infty c_m (R_m(y_l - \pi) - R_m(\tilde{y_l} - \pi)).
\end{align*}
Defining $\boldsymbol{\phi} \in \mathbb{R}^{L}$, $\boldsymbol{c} \in \mathbb{R}^\infty$, and $\boldsymbol{R} \in \mathbb{R}^{L \times \infty}$
by:
\begin{align*}
  \boldsymbol{\phi}_l = \phinear(\tilde{y}_l) - \phinear(y_l), \hspace{1cm} \boldsymbol{c}_m = c_m, \hspace{1cm} \boldsymbol{R}_{l,m} = R_m(y_l - \pi) - R_m(\tilde{y}_l - \pi)
\end{align*}
lets us write $\boldsymbol{\phi} = \boldsymbol{R}
\boldsymbol{c}$. Then, to apply the periodic summation method, we
recover $\boldsymbol{c}$ by solving a minimization problem and, for
$y$ such that $0 \leq y < 2\pi$, compute
$\phi(y) = \phinear(y) + \phifar(y)$, where $\phifar(y)$ is computed
from:
\begin{align*}
  \phifar(y) = \sum_{m=0}^\infty c_m R_m(y - \pi).
\end{align*}

%%% Local Variables:
%%% mode: latex
%%% TeX-master: "writeup.tex"
%%% End:
