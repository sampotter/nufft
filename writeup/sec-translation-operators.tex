\section{Cauchy Kernel Translation Operators}

Our algorithm makes use of the multilevel fast multipole method
(MLFMM) in its implementation. The MLFMM is developed in particular
for our kernel $\Phi$. In this section, we derive the corresponding
$\SSmat$, $\SRmat$, and $\RRmat$ translation operators. We also derive
two simple but useful results. The first shows that translation of a
truncated $R$ or $S$ factorization corresponds to multiplying the
truncated expansion coefficients by the matrix obtained by considering
the upper left block of the infinite matrix representation of the
translation operator. The second result extends this slightly to
consider translations of linear combinations of truncated
factorizations. The second result is very useful for implementing the
upward and downward passes of the MLFMM.\@

\subsection{$\SSmat$ Translation Operator}

For the $\SSmat$
operator, we fix $m \in \nats$
and $x_*, x_*' \in \mathbb{R}$,
defining $\delta$ as before. Then, considering $S_m$, we can write:
\begin{align*}
  S_m(y - x_*) = \parens{y - x_*' + \delta}^{-m-1} = \parens{y - x_*'}^{-m-1} \parens{1 + {\delta \over y - x_*'}}^{-m-1}.
\end{align*}
We have, for $|x| < 1$,
that the Taylor series expansion of $\parens{1 + x}^{-m-1}$
is given by:
\begin{align*}
  \parens{1 + x}^{-m-1} = \sum_{n=0}^\infty {-m-1 \choose n} x^n,
\end{align*}
where we treat ${-m-1 \choose n}$
as \emph{generalized} binomial coefficients. These are given by:
\begin{align*}
  {-m-1 \choose n} = \prod_{k=1}^n {-m-1-k+1 \over k} = \parens{-1}^n {(m+n)! \over m!n!}.
\end{align*}
Applying this form for $\parens{1 + x}^{-m-1}$
to our expression for $S_m$, we can write:
\begin{align*}
  S_m(y - x_*)
  &= \parens{y - x_*'}^{-m-1} \sum_{n=0}^\infty \parens{-1}^n {\parens{m + n}! \over m! n!} \parens{\delta \over y - x_*'}^n \\
  &= \sum_{n = 0}^\infty \parens{-1}^n {\parens{m + n}! \delta^n \over m! n!} S_{m+n}(y - x_*') \\
  &= \sum_{n = m}^\infty \parens{-1}^{n-m} {n! \delta^{n-m} \over \parens{n - m}! m!} S_n(y-x_*') & \mbox{(mapping $n \mapsto n - m$)}.
\end{align*}
Where, again, we require that $\abs{\delta} < \abs{y - x_*'}$,
from having made use of the Taylor series expansion for
$\parens{1 + x}^{-m-1}$.

If we consider $\SSmat$
to be an infinite matrix with the $(n,m)$th
entry given by the $n$th
coefficient of the preceding series expansion for $S_m$,
we can define:
\begin{align*}
  \parens{\SSmat}_{n,m} \equiv \parens{-1}^{n - m} {n! \delta^{n-m} \over \parens{n - m}! m!},
\end{align*}
which then allows us to write:
\begin{align*}
  S_m(y - x_*) = \sum_{n=0}^\infty \parens{\SSmat}_{n,m} S_n(y - x_*').
\end{align*}
Schematically, the matrix $\SSmat$
can be written as follows (bearing in mind that $m \geq 0$
and $n \geq m$, with entries equal to zero for $n < m$):
\begin{align*}
  \SSmat = \begin{bmatrix} 
    1 & 0 & 0 & 0 & 0 & \cdots \\
    -\delta & 1 & 0 & 0 & 0 & \cdots \\
    \delta^2 & -2\delta & 1 & 0 & 0 & \cdots \\
    -\delta^3 & 3\delta^2 & -3\delta & 1 & 0 & \cdots \\
    \delta^4 & -4\delta^3 & 6 \delta^2 & -4 \delta & 1 & \cdots \\
    \vdots & \vdots & \vdots & \vdots & \vdots & \ddots \\
  \end{bmatrix}
\end{align*}

\subsection{$\SRmat$ Translation Operator}

To determine the $\SRmat$
operator, we start by directly considering the Taylor series of $S_m$
expanded at $\delta$. We have:
\begin{align*}
  S_m(x) = \sum_{n=0}^\infty {S_m^{(n)}(\delta) \over n!} \parens{x - \delta},
\end{align*}
where the $n$th
derivative of $S_m$ evaluated at $\delta$ is computed to be:
\begin{align*}
  S_m^{(n)}(\delta) = \left. {d^n S_m \over dx^n} \right|_{x = \delta} = \parens{-1}^n \parens{m+1} \parens{m+2} \cdots \parens{m+n} \delta^{-m-n-1} = \parens{-1}^n {\parens{m + n}! \over m! \delta^{m+n+1}}.
\end{align*}
Then, substituting this expression into our Taylor series expansion for
$S_m$ and evaluating at $y - x_*$ gives us:
\begin{align*}
  S_m(y - x_*) = \sum_{n=0}^\infty {\parens{-1}^n \parens{m + n}! \over m! n! \delta^{m + n + 1}} \parens{y - x_* - \delta}^n = \sum_{n=0}^\infty {\parens{-1}^n \parens{m + n}! \over m! n! \delta^{m + n + 1}} R_n(y - x_*'),
\end{align*}
since $y - x_* - \delta = y - x_*'$.

Again, having expressed $S_m(y - x_*)$
as a series in $R_n(y - x_*')$,
we have a means of defining $\SRmat$
by reading off the coefficients of these series for each $m$.
This gives us:
\begin{align*}
  \parens{\SRmat}_{n,m} \equiv {\parens{-1}^n \parens{m + n}! \over m! n! \delta^{m+n+1}}.
\end{align*}
As before, we can use these coefficients to express $S_m$
as a series in $R_n$
by writing:
\begin{align*}
  S_m(y - x_*) = \sum_{n=0}^\infty \parens{\SRmat}_{n,m} R_n(y - x_*').
\end{align*}
The resultant infinite matrix $\SRmat$ can be depicted as follows:
\begin{align*}
  \SRmat = \begin{bmatrix}
    \delta^{-1} & \delta^{-2} & \delta^{-3} & \delta^{-4} & \cdots \\
    -\delta^{-2} & -2\delta^{-3} & -3\delta^{-4} & -4\delta^{-5} & \cdots \\
    \delta^{-3} & 3\delta^{-4} & 6 \delta^{-5} & 10 \delta^{-6} & \cdots \\
    -\delta^{-4} & -4 \delta^{-5} & -10 \delta^{-6} & -20 \delta^{-7} & \cdots \\
    \vdots & \vdots & \vdots & \vdots & \ddots
  \end{bmatrix}
\end{align*}
By considering this matrix, we can see that the antidiagonals are
constant in their power for $\delta$
and are multiplied by the binomial coefficients with alternating sign.

\subsection{$\RRmat$ Translation Operator}

The $\RRmat$
translation operator is a bit simpler. If we consider $R_m(y - x_*)$,
we can write:
\begin{align*}
  R_m(y - x_*) = (y - x_*)^m = (y - x_*' + x_*' - x_*)^m = (y - x_*' + \delta)^m.
\end{align*}
Then, applying the binomial formula, we find that:
\begin{align*}
  R_m(y - x_*) = \sum_{n=0}^m {m \choose n} \delta^{m - n} \parens{y - x_*'}^n = \sum_{n=0}^m {m! \delta^{m - n} \over (m - n)! n!} R_n(y - x_*').
\end{align*}
Then, reading off the coefficients of this expansion, we can define
the entries of $\RRmat$ by:
\begin{align*}
  \parens{\RRmat}_{n,m} = \left\{ \begin{tabular}{cl}
                                    ${m! \delta^{m - n} \over (m - n)! n!}$ & if $n \leq m$, \\
                                    $0$ & otherwise.
                                  \end{tabular} \right.
\end{align*}
This allows us to write our reexpansion compactly:
\begin{align*}
  R_m(y - x_*) = \sum_{n=0}^\infty \parens{\RRmat}_{n,m} R_n(y - x_*').
\end{align*}
Writing out $\RRmat$ explicitly gives us:
\begin{align*}
  \RRmat = \begin{bmatrix}
    1 & \delta & \delta^2 & \delta^3 & \delta^4 & \cdots \\
    0 & 1 & 2\delta & 3 \delta^2 & 4\delta^3 & \cdots \\
    0 & 0 & 1 & 3 \delta & 6\delta^2 & \cdots \\
    0 & 0 & 0 & 1 & 4 \delta & \cdots \\
    0 & 0 & 0 & 0 & 1 & \cdots \\
    \vdots & \vdots & \vdots & \vdots & \vdots & \ddots
  \end{bmatrix}
\end{align*} 

\emph{Note: this is the translation operator that differs from what
  was derived in lecture.}

\subsection{Application of the Translation Operators}

Focusing on the $\SSmat$
translation operator, we can consider the application of it to the
kernel $\Phi$
to reexpand $\Phi$
about a new expansion center $x_*'$.
Considering a singular expansion about $x_*$,
our kernel $\Phi$ can be written:
\begin{align*}
  \Phi(y, x) = \sum_{m=0}^\infty b_m(x, x_*) S_m(y - x_*).
\end{align*}
And, as before, the $\SSmat$
translation operator lets us expand each $S_m(y - x_*)$ as:
\begin{align*}
  S_m(y - x_*) = \sum_{n=0}^\infty \parens{\SSmat}_{n,m} S_n(y - x_*').
\end{align*}
So, if we substitute our reexpansion of $S_m$
into our expression for $\Phi$, we can compute:
\begin{align*}
  \Phi(y, x)
  &= \sum_{m=0}^\infty b_m(x, x_*) \sum_{n=0}^\infty \parens{\SSmat}_{n,m} S_n(y - x_*') \\
  &= \sum_{n=0}^\infty \parens{\sum_{m=0}^\infty \parens{\SSmat}_{n,m} b_m(x, x_*)} S_n(y - x_*').
\end{align*}

From this expression, can see that reexpansion of the kernel $\Phi$
about $x_*'$ is tantamount to multiplication of the coefficients $b_m$
by the infinite matrix $\SSmat$, with the sequence of coefficients
treated as an infinite column vector. Of course, the same sort of
result holds for the $\SRmat$ and $\RRmat$ translation operators.

\subsection{Translations of Linear Combinations}

A small result that will be useful in the construction of the
multilevel fast multipole method algorithm which we will make use of
is the following. Consider a set of source points $\set{x_k}$
and a set of weights $\set{u_k}$,
both of size $N$,
and a kernel $\Phi$
which is a linear combination of $N$ other kernels, defined by:
\begin{align*}
  \Phi(y) = \sum_{k=1}^N u_k \Phi(y, x_k).
\end{align*}
As usual, we let $x_*$
denote our original expansion center and $x_*'$
our new expansion center. Without loss of generality, we will consider
$S$
expansions of the kernels $\Phi(y, x_k)$
and the $\SSmat$
translation operator. Then, for a general argument $y$,
we can reexpress $\Phi(y)$ itself by writing:
\begin{align*}
  \Phi(y)
  &= \sum_{k=1}^N u_k \Phi(y, x_k) \\
  &= \sum_{k=1}^N u_k \sum_{m=0}^\infty b_m(x_k, x_*) S_m(y - x_*) \\
  &= \sum_{k=1}^N u_k \sum_{n=0}^\infty \parens{\sum_{m=0}^\infty \parens{\SSmat}_{n,m} b_m(x_k, x_*)} S_n(y - x_*') \\
  &= \sum_{n=0}^\infty \parens{\sum_{k=1}^N u_k \sum_{m=0}^\infty \parens{\SSmat}_{n,m} b_m(x_k, x_*)} S_n(y - x_*').
\end{align*}
From this final expression, we can see that translating linear
combinations of kernels that are expanded about the same expansion
center can be achieved by computing a new sequence of coefficients by
evaluating the same linear combination of the $\SSmat$-translated
coefficients. 

Extending this formulation slightly, separate original expansion
centers can also be handled straightforwardly. This will be useful,
for example, in aggregating multipole expansions in the ``upward
pass'' phase of the multilevel fast multipole method.

%%% Local Variables:
%%% mode: latex
%%% TeX-master: "writeup.tex"
%%% End:
