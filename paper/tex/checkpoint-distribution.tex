\section{Checkpoint Distribution}

The distribution of the checkpoints
$\phiargcp_0, \hdots, \phiargcp_{\numcps-1}$ can have a significant
effect on the accuracy of the periodic summation method. The Cauchy
kernel was found to be very poorly conditioned depending on the
checkpoint distribution. The decomposition
$\m{R} = \vandermondefactor\uppertrifactor$ reveals the reason---one of its
factors is a Vandermonde matrix, which is known to be very poorly
conditioned, in general. The determinant of $\vandermondefactor$ is given
by:
\begin{align}
  \det(\vandermondefactor) = \prod_{0 \leq \cpindex < \cpindex' < \numcps} (\fitarg_\cpindex - \fitarg_{\cpindex'}).
\end{align}
A necessary condition for the algorithm to work, then, is for the
checkpoints to be unique---if any are equal, $\vandermondefactor$ is
singular. However, beyond this, if the checkpoints are spread too far
apart, the determinant quickly becomes unbounded, resulting in a
poorly conditioned fitting matrix. As our matrix decomposition
requires us to fix $\persumindex$ for each $\phiargcp_\cpindex$, we simply let
$\persumindex = \pm 1$.

\begin{itemize}
\item \TODO\ add section that gets into the condition number of the
  fitting matrix, and talks about how to choose checkpoints based off
  of this.
\item \TODO\ add figure of wacky interpolated polynomials that are all
  screwed up.
\item \TODO\ compute actual determinant and do a little smarter
  analysis---determinant of $\uppertrifactor$ is just $\truncnum!$.
\item \TODO\ it's probably possible to bound the determinant (and also
  $\kappa$) by using the above two bits. Don't forget about inverse
  vs. normal matrix\ldots
\item \TODO\ does replacing $y_l$ with $z_l$ improve the condition
  number?
\item \TODO\ This might be a little wrong\ldots Determinants don't really
  seem to be related to the condition number. Look into some
  references for this.
\item \TODO\ Add plot of $\log_{10}(\kappa(\m{R}))$ (approximation of
  number of digits lost).
\end{itemize}

%%% Local Variables:
%%% mode: latex
%%% TeX-master: "../paper.tex"
%%% indent-tabs-mode: nil
%%% End:
