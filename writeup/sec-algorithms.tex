\section{Algorithms}

\subsection{MLFMM}

The periodic summation method makes use of a ``black box fast
summation algorithm''\----this could, for instance, be the usual
matrix product. In our case, as well as in the original paper, the
fast multipole method is used. There are many variations on the fast
multipole method~\cite{book:fmm-helmholtz}. Our own implementation is
not particularly refined, and is a nonadaptive MLFMM that takes both
the truncation number and the depth of the MLFMM to be computed as
parameters. More refined choices for each of these could be made; in
particular, making use of the adaptive MLFMM is likely a good
choice. As for the truncation number, were the adaptive MLFMM to be
adopted, this could be considered once that change were made.

\subsubsection*{FMM Input/Output Scaling}

A brief observation about the FMM: assume that $X$ and $Y$ are your
input sources and targets to a run of the FMM, and that they are
sorted but not necessarily normalized to lie in the interval $[0, 1]$.
It looks like it's necessary to scale in the following
fashion.
\begin{enumerate}
\item Compute $a \equiv \min(\min X, \min Y) $
  and $b \equiv \max(\max X, \max Y)$.
\item Set $\alpha \equiv b - a$ and $\delta \equiv -a$.
\item Compute $X'$ and $Y'$ from:
  \begin{align*}
    X' \equiv \set{(x + \delta)\alpha^{-1} : x \in X}, \hspace{1cm} Y' \equiv \set{(y + \delta) \alpha^{-1} : y \in Y}.
  \end{align*}
\item Evaluate the FMM with $X'$
  and $Y'$
    in place of $X$
    and $Y$ as arguments and multiply the results by $\alpha$.
\end{enumerate}
In a test case, this seemed to provide the necessary results.

\subsection*{Inverse Interpolation}

In our case, compared to the algorithm presented in the periodic
method summation paper, our algorithm is a bit simplified. The
corresponding algorithm for the inverse interpolation step to be computed during the inverse NUFFT is straightforward. \\

\begin{enumerate}
\item Choose $r, R, n,$ and $p$.
\item Determine the set of check points $Y$ with $|Y| = L > p$.
\item Compute $\boldsymbol{\phi}_{\operatorname{near}}$ using the ``scaled'' MLFMM.
\item Approximately compute $\bold{c}$ from $- \bold{R} \bold{c} =
  \boldsymbol{\phi}_{\operatorname{near}} + \boldsymbol{\epsilon}^{(p)}$.
\item For each source point $y_j$,
  evaluate $\phi(y_j) = \phinear(y_j) + \phifar(y_j)$.
\item Compute each interpolation point $g_j$ from
  Equation~\ref{eqn:gj-per-sum-form}.
\end{enumerate}

%%% Local Variables:
%%% mode: latex
%%% TeX-master: "writeup.tex"
%%% End:
