\section{Bandlimiting, the Fourier Series, and the DFT}

The point of this section is to make the meaning of ``let $f$ be
bandlimited'' a bit more precise in terms of the sampling theorem. We
let $f : \mathbb{R} \to \mathbb{C}$ be a $2\pi$-periodic function. The
Fourier series of $f$ has two equivalent formulations. The first is as
a trigonometric series of the following form:
\begin{align} \label{eqn-fourier-series-sinusoid-rep}
  f(x) = {a_0 \over 2} + \sum_{n=1}^\infty a_n \cos(n x) + \sum_{n=1}^\infty b_n \sin(n x)
\end{align}
where the coefficients are given by:
\begin{enumerate}
\item $a_0 = {1 \over \pi} \int_0^{2\pi} f(x) dx$.
\item $a_n = {1 \over \pi} \int_0^{2\pi} f(x) \cos(nx) dx$.
\item $b_n = {1 \over \pi} \int_0^{2\pi} f(x) \sin(nx) dx$.
\end{enumerate}
The second formulation is in terms of a series of weighted complex
exponentials:
\begin{align*}
  f(x) = \sum_{n = -\infty}^\infty c_n e^{inx}
\end{align*}
where the coefficients $c_n$ are given by:
\begin{align*}
  c_n = {1 \over 2} \left\{ \begin{tabular}{cl}
                              $a_n + ib_n$ & if $n < 0$, \\
                              $a_0$ & if $n = 0$, \\
                              $a_n - ib_n$ & if $n > 0$. \\
                            \end{tabular} \right.
\end{align*}
We are interested in the complex exponential form, in particular; but
we note that toy problems are easier to conceive of in terms of
sinusoids. So, we simply observe that the preceding formulae for the
different sequences of coefficients can be used to translate between
the two representations easily.

Now, assume that our function $f : \mathbb{R} \to \mathbb{C}$
is \emph{bandlimited}. We take this to mean that $f$'s
Fourier series is finite. That is, there exists some set
$\mathcal{N} \subseteq \mathbb{Z}$
with $|\mathcal{N}| < \infty$ such that:
\begin{align*}
  f(x) = \sum_{n \in \mathcal{N}} c_n e^{inx}.
\end{align*}
Of course, if we want, this condition can be reexpressed in terms of
Equation \ref{eqn-fourier-series-sinusoid-rep}.  We let
$n_0 = \max \set{\abs n : n \in \mathcal{N}}$. Then, if $K > n_0$, we
have that:
\begin{align} \label{eqn:zero-padded-fourier-series-of-f}
  \sum_{n \in \mathcal{N}} c_n e^{inx} = \sum_{n=-(K-1)}^{K-1} c_ne^{inx}.
\end{align}
The sampling theorem says that if we want to sample the function
defined by the right hand side of
Equation~\ref{eqn:zero-padded-fourier-series-of-f}, we must sample at
a rate at least twice that of the highest frequency. With the signal
unknown, although this step appears formal, it is imperative that we
choose a rate that is at least $2K$. So, assuming a sampling rate of
$2K$, we define $x_k$ by:
\begin{align*}
  x_k \equiv 2\pi k /(2K) = \pi k / K.
\end{align*}
Then, the sampling theorem implies that the coefficients of the
$2K$-point discrete fourier transform (DFT) agree with the
corresponding Fourier series coefficients. This lets us write:
\begin{align} \label{eqn:2K-point-dft-of-f}
  c_n = {1 \over 2K} \sum_{k=0}^{2K-1} f(x_k) e^{-inx_k}.
\end{align}
Substituting Equation~\ref{eqn:2K-point-dft-of-f} into the right hand
side of Equation~\ref{eqn:zero-padded-fourier-series-of-f} forms the
basis of the derivation of our algorithm. This substitution will be
carried out in the next section in detail.
